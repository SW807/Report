Dette projekt bygger videre på det fundament som blev udarbejdet i \cite{faelles}.
Det endelige resultat var en platform til Android, hvor adskillige moduler kan tilføjes for at danne en tilpasset applikation.

Der er mange mulighed  for indsamling og logning af data via mobile enheder \cite[Indsamling af data]{faelles}.
De faktorer som vurderes de vigtigste og mest effektive, baseret på de to person interviews, samt fokusgruppeinterviewet, er søvn og social aktivitet.
Den anden af de to grupper har valgt at fokusere på søvn, hvorfor dette projekt vil fokusere på det sociale.

Gennem interviewet med Jørgen Aagaard, blev det slået fast at det sociale spiller en væsentlig rolle ved detektering af episoder.
Ved depression mindskes det hvor meget der siges og skrives og aktiviteter vil blive nedsat.
Ved mani vil det være modsat, da der vil være en stigning i kontakt udadtil.
Det er derved interessant at måle og observere ændringer i mængden af social kontakt.

For at være i stand til at kunne måle på disse ændringer, skal der fastslås en model og derudfra en baseline.
Herefter bliver det bliver muligt at sammenligne målt adfærd med tidligere adfærd, og derved observere ændringer.

Det har dog ikke været muligt at finde et godt grundlag for denne model, hvilket har resulteret i at resten af dette projekt vil fokusere på at analysere på forbindelsen mellem det sociale og sindstilstand, samt at danne en ny platform til at indsamle data til at specificere denne forbindelse.