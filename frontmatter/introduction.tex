Dette projekt bygger videre på det fundament som blev udarbejdet i starten af semestret \citefaelles{}.
Denne rapports resultat var en platform til Android, hvor en kombination af moduler kan anvendes til at danne en brugertilpasset applikation.

De faktorer som blev vurderet de vigtigste og mest effektive, baseret på de to personinterviews samt fokusgruppeinterviewet, er søvn og social aktivitet.
De to projektgrupper har derfor besluttet at fokusere på hver sin af disse faktorer og udarbejde løsninger hertil.
Dette projekt vil fokusere på \emph{social aktivitet}.

Gennem interviewet \citefaelles{Afsnit 1.4}~ gav Jørgen Aagaard udtryk for at social aktivitet spiller en væsentlig rolle ved detektering af episoder.
Under en depression mindskes den sociale aktivitet mens den øges i en manisk periode.
Det vil derfor være interessant at måle og observere ændringer i mængden af social kontakt.
Da en del af en persons sociale kontakt foregår via mobiltelefonen (i form af sms-beskeder og opkald) vil det være muligt at observere en del af en persons sociale kontakt ved brug af den udviklede platform.
Tilsvarende kan information indsamles fra \textit{sociale applikationer} som fx. Facebook og Twitter.

Det har ikke været muligt for projektgruppen at finde dokumentation for relationen mellem social adfærd og stemningsleje hos den enkelte patient.
Af denne grund har det ikke været muligt at opstille en model for relationen.
Projektgruppen har i stedet arbejdet med at muliggøre psykiatrisk forskning, der ved indsamling af stemningslejet hos patienter, samt patientens digitale sociale aktivitet, kan være med til at beskrive relationen mellem de to aspekter (hvis en sådan eksisterer).
