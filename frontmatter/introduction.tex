Dette projekt bygger videre på det fundament som blev udarbejdet i starten af semestret \citefaelles{}.
Det endelige resultat her var en platform til Android, hvor en kombination af adskillige moduler kan anvendes for at danne en brugertilpasset applikation.

Der er mange muligheder  for indsamling og logning af data via mobile enheder \citefaelles{Indsamling af data}.
De faktorer som vurderes de vigtigste og mest effektive, baseret på de to personinterviews samt fokusgruppeinterviewet, er søvn og social aktivitet.
De to projektgrupper har derfor besluttet at fokusere på hver sin af disse faktorer og udarbejde løsninger hertil.
Dette projekt vil derfor fokusere på social aktivitet.

Gennem interviewet\mikkel{Indsæt interview citation - jeg er ikke selv sikker på hvilken citation jeg skulle have indsat} gav Jørgen Aagaard udtryk for, at social aktivitet spiller en væsentlig rolle ved detektering af episoder.
Under en depression mindskes den sociale aktivitet mens den øges i en manisk periode.
Det vil derfor være interessant at måle og observere ændringer i mængden af social kontakt.
Da en del af en persons sociale kontakt foregår via mobiltelefonen (i form af sms-beskeder og opkald) vil det være muligt at observere en del af en persons sociale kontakt ved brug af den udviklede platform.
Tilsvarende kan information indsamles fra \textit{sociale applikationer} som fx Facebook og Twitter.

Det har ikke været muligt for projektgruppen at finde dokumentation for relationen mellem social adfærd og stemningsleje hos den enkelte patient.
Af denne grund har det ikke været muligt at opstille en model for relationen.
Projektgruppen har i stedet arbejdet med at muliggøre psykiatrisk forskning, der ved indsamling af stemningslejet hos patienter samt patientens digitale sociale aktivitet kan være med til at beskrive relationen mellem de to aspekter (hvis en sådan eksisterer).

For at kunne opnå dette mål implementeres et modul på den eksisterende platform \citefaelles{} der tillader at forespørge den enkelte patient om dennes stemningsleje.
Modulet søges så fleksibelt at det vil kunne understøtte forskellige spørgsmål og dermed undersøgelse af relation mellem en række problemstillinger.

Mulighederne inden for opsamling af data om patientens digitale sociale adfærd vil også blive undersøgt og et antal moduler vil blive implementeret.

Det er projektgruppens mål at muliggøre forskning som beskrevet herover, men ikke udføre disse.
Udførelsen af denne forskning vil kræve indsamling af data over en længere periode for derefter at kunne analyseres for sammenhænge, samt inklusionen af en psykolog.