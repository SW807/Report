This project extends on the 'PsyLog - mobile platform'\cite{faelles}.
The purpose of this project is to prepare for an experiment which examines if there is a connection between non-intrusive and intrusive data collected from patients, that have affective disorder, mobile phone regarding to their social activity on their mobile phone.

The project examines the options for collecting non-intrusive data from the mobile platform Android and has also implemented two of those; \textit{calls} and \textit{SMS-MMS}.

Further the project examines the options for collecting intrusive data from patients, but this is still a small research area.
This means the project had to examine methods used in psychology for determining the mental state of a patient.
The chosen methods were analysed and some needs were defined for having a satisfied solution for collecting intrusive data.
The project implements a module for the 'PsyLog - mobile platform' that can be used to construct the methods for collecting th intrusive data.

By collecting data from both, intrusive and non-intrusive, it should be possible in the future to say something about the non-intrusive data.
\bruno{Første udkast til abstract}