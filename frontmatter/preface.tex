Denne rapport er udarbejdet af Software-Ingeniør studerende på 8. semester, på Aalborg Universitet, forårssemestret 2015.
Det forventes at læseren har en baggrund indenfor IT/software grundet det tekniske indhold.

Rapporten er anden del af rapporten ``''\stefan{indsæt endelig titel} udarbejdet i et samarbejde mellem SW807F15 og SW808F15.
Den første del af rapporten omhandler den indledende analyse af affektive lidelser samt udviklingen af en platform der kan understøtte patienter med affektive lidelser.
Denne del af rapporten vil omhandle udviklingen af et konkret modul til den udviklede platform.
\mikkel{Er det ikke to forskellige rapporter?
Det giver ikke så meget mening at citere fælles-rapporten hvis den bare er en del af rapporten.}

Hvert kapitel starter med en introduktion, hvor indhold og udførelse kort belyses.
Yderligere vil der, hvor det er relevant, blive givet en opsummering af kapitlet.

Referencer og citationer noteres med nummernotation, fx \cite{faelles}, hvilket refererer til den første kilde i litteraturlisten.
Når der refereres til 'vi', menes der projektgruppens medlemmer.
\newline
\newline
Vi vil gerne sige tak til vores undervisere, samt vores vejleder Ivan Aaen for et godt samarbejde samt god vejledning igennem projektforløbet.