Denne rapport er udarbejdet af Software-Ingeniør studerende på 8. semester, på Aalborg Universitet, forårssemestret 2015.
Det forventes at læseren har en baggrund indenfor IT/software grundet det tekniske indhold.

Dette projekt er en del af et fælles-projekt ``PsyLog - En Modulær Mobil Platform med Fokus på Affektive Lidelser''\cite{faelles} udarbejdet i et samarbejde mellem SW807F15 og SW808F15.
I fællesskab er der udarbejdet en rapport der omhandler den indledende analyse af affektive lidelser, samt udviklingen af en platform der kan understøtte patienter med affektive lidelser.
Denne rapport er udarbejdet af SW807F15 og vil omhandle udviklingen af konkrete moduler til den udviklede platform.

Hvert kapitel starter med en introduktion, hvor indhold og udførelse kort belyses.
Yderligere vil der, hvor det er relevant, blive givet en opsummering af kapitlet.

Referencer og citationer noteres med nummernotation, fx. \cite{faelles}, hvilket refererer til den første kilde i litteraturlisten.
Når der refereres til 'vi', menes der projektgruppens medlemmer.
\newline
\newline
Vi vil gerne sige tak til vores undervisere, samt vores vejleder Ivan Aaen, hvor sidstnævnte skal særligt takkes for et godt samarbejde og god vejledning igennem projektforløbet.
