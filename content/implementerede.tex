\section{Implementerede Moduler}
I \cref{datasamling} blev en række mulige datakilder beskrevet for at klargøre de muligheder projektgruppen så for indsamling af information der relaterer sig til en patients sociale aktivitet.
Af disse muligheder er indsamlingen af opkalds-oplysninger og meta information om sms og mms beskeder blevet implementeret.
Disse er udvalgt, da de repræsenterer den traditionelle brug af en mobiltelefon samt en simpel tilgang til informationen.

Da dette projekt søger at etablere en platform til at foretage eksperimenter der skal hjælpe med at klarlægge en relation mellem social aktivitet og sindstilstand, er kun de simpleste informationskilder blevet implementeret.
Det betyder at tiden bedre har kunnet udnyttes til implementation af den underliggende platform.

De to informationskilder er implementeret som to separate sensor moduler\mikkel{Mulig reference til fælles rapport.}, der hver især indsamler meta-information om deres respektive kilder.

Begge moduler er implementeret således at de, hver gang de startes, indlæser alt nytilkommen data.
Denne opbygning tillader modulerne at være konstrueret uafhængigt af endeligt brug.
Derved laves der under udvikling af modulerne færre restriktioner på deres brug i den endelige løsning.
Hvor ofte denne opdatering finder sted er givet ved hvert af modulernes modul-definitioner \citefaelles{4.4 Modul-definition}.

De to moduler, det data de hver indsamler samt deres repræsentation af dette data beskrives i større detalje herunder.
%Argumentation ud fra dataindsamlingsmuligheder
%opkald
%sms/mms

\paragraph{Opkald}
Ud fra beskrivelsen givet i \cref{datasamling:opkald}, er der ikke mange muligheder for indsamling af data for opkald.
Alt nødvendig information opbevares af Android i en enkelt database tabel.
Der kan dermed foretages en direkte mapping fra denne tabel, til modulets egen tabel, ved blot at kopiere de aktuelle informationer.

Modul-definitionen for opkald kan ses i \cref{module:calls}.
Af denne fremgår det at de information der indhentes om hvert opkald er:
\begin{itemize}
\item Kontakt oplysning (telefonnummer).
\item Dato for opkaldet.
\item Opkaldets varighed.
\item Hvorvidt opkaldet var indgående/udgående.
\item Hvorvidt opkaldet blev besvares eller ej.
\end{itemize}

Ved at kopiere disse informationer, tillades en ensartet tilgang til dem for andre moduler og der skabes en uafhængighed af historikkens størrelse.
Det bliver altså på denne måde muligt at styre hvor længe informationen skal være tilgængelig for andre moduler.

\begin{lstlisting}[float,
basicstyle=\ttfamily\footnotesize,
caption={Modul definitionen for opkald},label=module:calls]
{
  "name": "module_calls_meta",
  "_version": 1.0,
  "tables": [
    {
      "name": "call_history",
      "columns": [
        { "name": "caller", "dataType": "TEXT", "_unit": "Caller number." },
        { "name": "date", "dataType": "INTEGER", "_unit": "Timestamp." },
        { "name": "length", "dataType": "REAL", "_unit": "Duration in seconds."  },
        { "name": "incoming", "dataType": "INTEGER", "_unit": "boolean", "_info": "True if the call was incoming." },
        { "name": "answered", "dataType": "INTEGER", "_unit": "boolean", "_info": "True if the call was answered." }
      ]
    }
  ],
  "task": {
    "type": "scheduled",
    "value": "03:00"
  }
}
\end{lstlisting}

\paragraph{SMS/MMS beskeder}
\begin{lstlisting}[float,
basicstyle=\ttfamily\footnotesize,
caption={Modul definitionen for sms/mms},label=module:sms]
{
  "name": "smsmmsmeta",
  "_version": 1.0,
  "tables": [
    {
      "name": "smsmmsmeta",
      "columns": [
        { "name": "contact", "dataType": "TEXT", "_unit": "Caller number" },
        { "name": "date", "dataType": "INTEGER", "_unit": "Timestamp" },
        { "name": "length", "dataType": "INTEGER", "_unit": "Number of characters" },
        { "name": "incoming", "dataType": "INTEGER", "_unit": "boolean", "_info": "True if the message is in an inbox; otherwise false." },
        { "name": "mms", "dataType": "INTEGER", "_unit": "boolean", "_info": "True if the message is an mms; otherwise false." }
      ]
    }
  ],
  "task": {
    "type": "scheduled",
    "value": "03:00"
  }
}
\end{lstlisting}
