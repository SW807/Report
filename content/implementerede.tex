\section{Implementerede Moduler}
I \cref{datasamling} blev en række mulige datakilder beskrevet for at klargøre de muligheder projektgruppen så for indsamling af information der relaterer sig til en patients sociale aktivitet.
Af disse muligheder er indsamlingen af opkalds-oplysninger og meta information om sms og mms beskeder blevet implementeret.
Disse er udvalgt, da de repræsenterer den traditionelle brug af en mobiltelefon samt en simpel tilgang til informationen.

Da dette projekt søger at etablere en platform til at foretage eksperimenter der skal hjælpe med at klarlægge en relation mellem social aktivitet og sindstilstand, er kun de simpleste informationskilder blevet implementeret.
Det betyder at tiden bedre har kunnet udnyttes til implementation af den underliggende platform.

De to informationskilder er implementeret som to separate sensor moduler\mikkel{Mulig reference til fælles rapport.}, der hver især indsamler meta-information om deres respektive kilder.

Begge moduler er implementeret således at de, hver gang de startes, indlæser alt nytilkommen data.
Denne opbygning tillader modulerne at være konstrueret uafhængigt af endeligt brug.
Derved laves der under udvikling af modulerne færre restriktioner på deres brug i den endelige løsning.
Hvor ofte denne opdatering finder sted er givet ved hvert af modulernes modul-definitioner \citefaelles{4.4 Modul-definition}.

De to moduler, det data de hver indsamler samt deres repræsentation af dette data beskrives i større detalje herunder.
%Argumentation ud fra dataindsamlingsmuligheder
%opkald
%sms/mms

\paragraph{Opkald}
Ud fra beskrivelsen givet i \cref{datasamling:opkald}, er der ikke mange muligheder for indsamling af data for opkald.
Alt nødvendig information opbevares af Android i en enkelt database tabel.
Der kan dermed foretages en direkte mapping fra denne tabel, til modulets egen tabel.
Modul-definitionen for opkald kan ses i \cref{module:calls}.

\input{content/modul_definitioner/calls}

\paragraph{SMS/MMS beskeder}
\section{SMS/MMS beskeder}
Ligesom opkalds-historik, er der i Android mulighed for adgang til SMS/MMS beskeder.
Her kan man få adgangs til lister af beskeder\footnote{Indbakke, udbakke, sendte beskeder etc.}, med for hver besked afsender/modtager, indhold (og derudaf længden) samt afsendt/modtaget tidspunkt.

Observation af beskeder vil fungere i høj grad på samme måde som opkald.
Det betyder at der heller ikke med beskeder fokuseres på kommunikationens indhold men i stedet på dens volumen.

\paragraph{Begrænsninger}
Da indsamlingen af besked-information i høj grad ligner indsamlingen af opkalds-information gør de samme begrænsninger sig også gældende her.
