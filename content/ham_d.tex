\subsection{Hamilton Depressionsskala(HAM-D)}
Psykolog Janne Rasmussen fortalte at man brugte HAM-D til at afgøre om hvilket depressions stadie man er på.\cite[Afsnit 1.3, Møde med Psykolog Janne Rasmussen]{faelles}
Derfor har vi valgt at kigge på HAM-D som en af metoderne til at måle stemningslejet.

HAM-D er en metode, af \citet{ham_d}, der består af 17 variable faktorer.
Disse variable faktorer giver intervieweren en værdi på 0-4 eller 0-2 alt efter hvor meget patienten har symptomet.
For at gøre det nemmere for intervieweren er de variable faktorer oftest omformuleret til 17 spørgsmål, den danske version kan findes på \citet{ham_d_dansk}.
Når alle værdier er afgivet adderes disse:
\begin{itemize}
	\item \textbf{0-13} ingen depression.
	\item \textbf{13-17} let depression.
	\item \textbf{18-24} moderat depression.
	\item \textbf{25-52} svær depression.
\end{itemize}

\paragraph{De 17 spørgsmål}

\bruno{Her skal der være en beskrivelse af HAM-D metoden og hvad man får ud af den.}