\subsection{Hamilton Depressionsskala(HAM-D)}
Psykolog Janne Rasmussen fortalte under interviewet at man brugte HAM-D til at afgøre om hvilket depressions stadie man er på.\cite[Afsnit 1.3, Møde med Psykolog Janne Rasmussen]{faelles}
Derfor er HAM-D valgt som en af metoderne til at måle stemningslejet.

HAM-D er en metode, af \citet{ham_d}, der kan bruges på patienter der tidligere har fået diagnosticeret depression. 
Metoden består af 17 variable faktorer.
Disse variable faktorer giver intervieweren en værdi på 0-4 eller 0-2\footnote{Kommer an på om det er muligt at have et interval på fem eller tre på en given variable faktor.} alt efter hvor meget patienten har symptomet(det maksimale antal point der kan opnås er 52 point).
Intervieweren er en faglig kompetent person indenfor området.
For at gøre det nemmere for intervieweren er de variable faktorer oftest omformuleret til 17 spørgsmål, den danske version kan findes på \citet{ham_d_dansk}.
Når alle værdier er afgivet adderes disse og siger noget om patientens stemningsleje:
\begin{itemize}
	\item \textbf{0-13} ingen depression.
	\item \textbf{13-17} let depression.
	\item \textbf{18-24} moderat depression.
	\item \textbf{25-52} svær depression.
\end{itemize}

Som nævnt foretages HAM-D af en interviewer der er faglig kompetent.
Da spørgsmålene skal præsenteres på en mobil platform lægger det op til følgende åbne problemstilling: \textit{Hvordan præsenteres HAM-D på en mobil platform uden en interviewer?}
\bruno{\textbf{Til Ivan:} Er det ok?}