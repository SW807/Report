\chapter{Psykiatrisk forskning}\label{problem}
For at muliggøre forskning i relationen mellem patientens stemningsleje og social aktivitet er det nødvendigt at indsamle informationer til disse to aspekter.
Den grundlæggende platform muliggør implementationen af data moduler \citefaelles{4.3.3 Moduler} til at foretage denne indsamling af informationer.
Dermed ønskes en implementation af et eller flere moduler der indikerer social aktivitet og stemningsleje.

\paragraph{Social aktivitet}
Modulerne til monitorering af social aktivitet kunne implementeres som analyse moduler, der opsummerer data fra flere kilder, for derved at give et håndgribeligt billede af en patients sociale aktivitet.
Men da der også i forhold til relationen mellem brug af mobiltelefon og social aktivitet er tale om grundlæggende forskning, foretages i stedet udelukkende implementationer af data moduler.
Herved gøres der ingen antagelser om hvilke typer information er de mest relevante i forskningsøjemed.
Dog vælges det, af tidshensyn, at kun få moduler implementeres.
Disse er beskrevet i \cref{implementerede_moduler}.

\paragraph{Stemningsleje}
Som kilde til bestemmelse af en patients stemningsleje ønskes det at benytte en automatiseret vurdering baseret på patientens besvarelse af en mængde spørgsmål.
Hvilke spørgsmål der præcis bør anvendes er uvist.
I \cref{maaling_af_stemningsleje} gennemgås en række metoder der danner grundlag for hvilke typer spørgsmål der bør understøttes.
På denne måde kan eksisterende viden om indsamling af stemningsleje overføres til det nye system.
For at kunne præsentere disse spørgsmål for patienten, implementeres et enkelt data modul på den eksisterende platform\citefaelles{} der administrerer spørgsmålene (af alle typer) samt hvornår patienten skal stilles hvert af disse.

Modulet søges så fleksibelt at det vil kunne understøtte en række forskellige spørgsmåls-typer.
Herved vil det kunne anvendes i et bredere perspektiv end blot til bestemmelse af stemningsleje.
Det vil sige at modulet bliver anvendeligt til undersøgelse af relation mellem en række problemstillinger samt til indsamling af informationer til patienten på niveau med andre data moduler.
Altså vil patienten kunne anvende modulet til at se historik for tidligere besvarelser.

Det er projektgruppens mål at muliggøre forskning som beskrevet herover, men ikke udføre denne.
Udførelsen af forskningen vil kræve indsamling af data over en længere periode for derefter at kunne analyseres for sammenhænge, samt inklusionen af en psykolog.

\section{Mål for projektet}
Herunder listes de mål der af projektgruppen er opstillet for projektet.
De listede mål anses som essentielle for at muliggøre udførelsen af eksperimenter som beskrevet herover.
\begin{itemize}
\item Implementation af et eller flere ikke-forstyrrende\footnote{Se \cref{begreber::ikke_forstyrrende}} data moduler til indsamling af information relateret til patientens sociale aktivitet.
\item Implementation af et forstyrrende\footnote{Se \cref{begreber::forstyrrende}} modul der kan præsentere spørgsmål for patienten.
\begin{itemize}
\item Modulet skal understøtte de spørgsmåls-typer der beskrives i \cref{maaling_af_stemningsleje}.
\item Modulet skal stå for administrationen af hvornår patienten skal præsenteres for det enkelte spørgsmål.
\end{itemize}
\item Diskussion af de problemstillinger der er forbundet med udførelsen af eksperimenter ved hjælp af de udviklede moduler.
\end{itemize}