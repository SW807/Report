\section{Survey - et modul til forstyrrende dataindsamling}
For at kunne udlede en sammenhæng mellem stemningsleje og social aktivitet skal der udover målinger af den sociale aktivitet også bruges forstyrrende indsamlet data fra patienten.
Til dette formål skal der konstrueres et modul der kan stille spørgsmål til patienten på varierende tidspunkter.
I \cref{krav::typer} blev de nødvendige spørgsmålstyper udledt fra den relevante forskning undersøgt i \cref{maaling_af_stemningsleje}.
Implementationen af disse vil blive beskrevet i det følgende.

\subsection{Implementation af spørgsmålstyper}
\stefan{debat: spørgsmålstyper, svartyper, noget tredje?}
Gennem dette afsnit vil der blive refereret til \cref{stemreg::spoergsmaal}, som er en eksempel implementation af metoden  \emph{stemningsregistrering} beskrevet i \cref{stemningsleje::stemningsregistrering}.
De tre typer af spørgsmål, beskrevet i \cref{krav::typer}, er blevet implementeret.
Eksempler på brug kan ses på \cref{stemreg::spoergsmaal}.
På \cref{stemreg::stemningsleje} kan ses en multiple choice spørgsmål hvor der kan vælges flere svarmuligheder.
Dette spørgsmål findes også i en variant hvor man kun kan vælge en mulighed, indikeret ved runde svarbokse (radio buttons).

\Cref{stemreg::soevn} viser et spørgsmål der besvares ved en skala. 
Spørgsmålene har to labels et for minimum og et for maksimum, samt en indikator der viser den valgte værdi.

\Cref{stemreg::notater_vaegt} viser et spørgsmål der skal besvares ved en tekst. 


\begin{figure}
	\centering
	\begin{subfigure}[b]{0.45\textwidth}
		\includegraphics[width=\textwidth]{stemningsregistrering_eksempel/stemningsleje}
		\caption{Multiple choice}\label{stemreg::stemningsleje}
	\end{subfigure}
	\hfill
\begin{minipage}[b]{0.45\textwidth}
	\begin{subfigure}[b]{\textwidth}
		\includegraphics[width=\textwidth]{stemningsregistrering_eksempel/soevn}
		\caption{Skala}\label{stemreg::soevn}
	\end{subfigure}
	\newline 
	\begin{subfigure}[b]{\textwidth}
		\includegraphics[width=\textwidth]{stemningsregistrering_eksempel/notater_vaegt}
		\caption{Tekst}\label{stemreg::notater_vaegt}
	\end{subfigure}	
\end{minipage}	
	\caption{Tre af de fem spørgsmål der bruges i metoden: stemningsregistrering( \cref{stemningsleje::stemningsregistrering}), for at vise de tre spørgsmåls typer.}\label{stemreg::spoergsmaal}
\end{figure}

\subsection{Scheduler}
For at kunne stille spørgsmål så fleksibelt som muligt er der blevet konstrueret en scheduler.
Stemningsregistrering og dagbog skal ske én gang om dagen, mens PANAS spørgsmål sagtens kan spredes ud over dagen for at få et bredere billede af patientens tilstand.


\paragraph{Notifikationer}
For ikke at forstyrre brugerens hverdag for meget bruger vi en notifikation til at advisere om et forestående spørgsmål.
Notifikationen indeholder spørgsmålsteksten, så patienten kan vurdere om han har tid til at besvare spørgsmålet.
Eksempler på disse notifikationer kan ses på \cref{noti}.

\begin{figure}
	\centering
	\includegraphics[width=0.6\textwidth]{stemningsregistrering_eksempel/notification_bar}
	\caption{Fire notifikationer sendt af Survey}\label{noti}
\end{figure}

Når der trykkes på notifikationen dukker spørgsmålet op i en dialogboks hvor patientes svar kan indføres.

For at gøre systemet endnu mere fleksibelt skal der være mulighed for at kunne udskyde et spørgsmål.
På denne måde har patienten fuld kontrol over hvornår han svarer, uden at være nødt til at skulle afbryde det han er i gang med.
Dette er endnu ikke implementeret.