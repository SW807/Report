\section{Positive Affect and Negative Affect Scales (PANAS)}
PANAS er en anerkendt liste af begreber relateret til positiv og negativ affekt, som bruges til at vurdere stemningsleje, som beskrevet i \citet{panas}.

% Baseret på tidligere skalaer, bevist uafhængighed begreberne mellem
% Afprøvet præcision med forskellige tidsintervaller
% Afprøvet præcision vha. test/gentest
% Afprøvet som måling af angst, depression og generelt psykologisk lidelse; God correlation ift. NA og kendte skalaer (dårlig, som forventet, correlation mellem PA og samme skalaer)

PANAS er et forsøg på at lave et skema, med skalaer til positiv og negativ affekt, til brug i forbindelse med estimering af humør.
Den er baseret på tidligere eksisterende skalaer, med et ønske om at lave en mere generel og pålidelig skala.
Dette er gjort ved at udvælge og afprøve en række begreber, hvis pålidelighed, præcision og uafhængighed er sikret gennem en lang række analyser og forsøg.

PANAS skemaet er 2 lister af henholdsvis 10 positive og 10 negative ord (oversat fra \citet{panas}):\\
\begin{center}
\begin{tabular}{r | l}
interesseret & irritabel \\\newline
aktiv & fjendtlig \\\newline
eksalteret & skamfuld \\\newline
inspireret & oprevet \\\newline
stærk & nervøs \\\newline
målbevidst & skyldig \\\newline
opmærksom & bange \\\newline
agtpågivende & anspændt \\\newline
entusiastisk & nødstedt \\\newline
stolt & ræd
\end{tabular}
\end{center}

Den komplette PANAS metode kan findes i \cref{panas:apppendix}.

Alle tyve vurderes med en værdi fra et til fem som indikerer i hvor høj grad begrebet er oplevet.
Om hvorvidt det er oplevet skal tænkes i forhold til en bestemt tidsperiode; \textit{I øjeblikket, I dag, De seneste par dage, De seneste par uger, Det seneste år, Generelt}.

Præcisionen af skalaerne er afprøvet først og fremmest ved at udfylde skemaet med blik på forskellige tidsperioder nævnt ovenover.
Yderligere for at verificere præcisionen, er skemaet afprøvet med samme person, samme tidsperiode(r) flere gange over en periode på femten uger, hvorefter de besvarelserne sammenlignes med en forventning om jo længere tidsperiode jo bedre sammenhold.

Vigtigst af alt er PANAS også blevet sammenlignet med kendte skemaer der bruges i forbindelse med måling af angst, depression og generel psykisk lidelse.
Her vises god korrelation mellem de skemaer og målene for negative affekt for PANAS.
Og omvendt, som forventet, vises dårlig korrelation mellem skemaerne og målene for positiv affekt.