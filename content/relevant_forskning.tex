% % % OVERORDNET
% Begreb
% Metode
% Metrik
% Hypotese

% % % ARTIKEL (Mood -> Sociability)
% Hypotese: Positivt humør fører til sociale aktiviteter
% Tidligere: sociale aktiviteter -> affekt
% Metode: Inducering af humør, vurdering af humør, vurdering af fundne aktiviteter

\chapter{Relevant forskning}
Ved at undersøge den nuværende forskning indenfor psykologi-faget, vil vi først og fremmest fastslå at der er en forbindelse mellem humør og social aktivitet.
Desuden fastslås også måden hvorpå dataindsamling og eksperimentering foregår indenfor dette fag, hvilket er ved interview og spørgeskemaer.
Dette mener vi at ville kunne gøres bedre med mere objektive data som supplement, som fx ved løbende forespørgsler via smartphone.
Derfor vil vi til sidst undersøge relevant forskning indenfor dataindsamling på smartphone, specifikt ift. stemningsleje.

\section{Sammenhæng mellem humør og social aktivitet}
Ud fra de definitioner der er givet i forrige afsnit, er det humør som vi er interesseret i, da det dækker over længerevarende ændringer, hvilket er tilfældet ved mani-depressive perioder.

For at undersøge den psykologiske forskning og eksperimentering der er foregået op til nu, i forhold til sammenhæng mellem humør og social aktivitet, tages der udgangspunkt i \citet{whelan}.

Det fastslås i \citet{whelan} at der er meget forskning der dokumenterer en 2-vejs forbindelse mellem affekt og social aktivitet\footnote{Engelsk: \textit{sociability}}.
Dog har der været begrænset med eksperimentering til at fastslå at godt humør, eller positiv affekt, fører til mere social aktivitet, hvilket er hypotesen for eksperimentet i artiklen.
Ud fra en undersøgelse af lignende artikler, her tages fat i artikler fra 1986 til selve artiklens publikation (2012), kritiseres den relaterede forskning.
De kritiserer nogle af disse tidligere artikler for at komme frem med en tvetydig retning, altså at det ikke vides om humør påvirker social aktivitet eller omvendt.
Derudover er der også kritik til de tidligere brugte metoder, som fejler i at skelne mellem behagelige og sociale aktiviteter.
Dvs. at der ikke er en klar skelning med de 4 kombinationer; behagelige sociale, ubehagelige sociale, behagelige ikke-sociale og ubehagelige ikke-sociale situationer.

\paragraph{Selve eksperimentet} 
Eksperimentet er delt op i 3 dele.
Første del består i at fastslå aktiviteter med varierende grad af socialitet og behagelighed.
Dette gøres online ud fra en gruppe af 58 studerende, hvor en liste på 82 aktiviteter vurderes på behagelighed og socialitet på en skala fra 1 til 5.
Dette ender ud i en liste af 28 aktiviteter, med en jævn fordeling af behagelighed.

De 2 næste dele er opbygget på samme måde, hvor henholdsvis 138 og 104 psykologi-studerende først ser et video-klip af positiv, neutral eller negativ natur.
Herefter, for at fastslå det nuværende humør, vurderes en række ord, baseret på PANAS (se \cref{panas}), mellem 1 og 7 for i hvor høj grad de opleves i øjeblikket.
Sidste trin er en evaluering af de 28 tidligere fastslåede aktiviteter; hvor de alle på en skala fra 1 til 7 vurderes hvor meget test-personen har lyst til at udføre aktiviteten i øjeblikket.

Det konkluderes ud fra eksperimentet at positivt humør fører til højere lyst til at udføre mere sociale aktiviteter.
Derudover noteres det også at der var en bemærkelsesværdig tendens for de personer der fik induceret negativt humør, havde mere lyst til ikke-sociale aktiviteter.

\section{Brug af mobil data-indsamling}
Der er fundet to eksempler på brugen af mobil data-indsamling relateret til detektering af humør-ændringer; \citet{social_sensing} og \citet{social_sensing_2}.

\subsection{Social Sensing for Epidemiological Behaviour Change}
% Metode: knytning mellem spørgeskema og ren mobil logning
% Knytter kun de to for 48 timers vinduer, ikke ændring over tid
% For tæt knyttet på fysisk sygdom, så resultater ikke de bedste for psykisk (kun 2 daglige meget store spørgsmål)
I \citet{social_sensing} bruges smartphone data til at undersøge mulighederne for at bruge mobil dataindsamling til at detektere adfærdsændringer, hvilket kan sige noget om en person er syg (fysisk eller psykisk).
Social data indsamles via opkaldshistorik, SMS-historik og WLAN-baseret lokationsestimering.
Bluetooth proximity bruges til at tælle ansigt-til-ansigt interaktioner hvor det er muligt. 
Derudover blev der hver dag udfyldt et spørgeskema, bestående af seks spørgsmål angående fysisk og psykisk helbred.

Forsøget blev udført med 70 beboere på et afsnit af et kollegium for bachelor-studerende, hvor de 70 beboere var ca. 80 \% af det totale antal beboere i afsnittet.
Data blev indsamlet fra 1. februar til 15. april (2.5 måned), hvor mobil data blev uploadet døgnet rundt og spørgeskemaer udfyldt hver morgen kl. 6.
Perioden blev delt op i 48 timers intervaller, hvor de spørgeskemaer der blev indsamlet i hvert interval blev knyttet til den indsamlet data for samme interval.

Resultaterne er i mindre grad interessante, da de er nærmere relateret fysisk sygdom og mildere depressioner.
Dog er det værd at bemærke at der blev fundet en sammenhæng mellem antallet af opdagede Bluetooth enheder, WIFI RFIDs, sociale interaktioner og fysisk/psykisk sygdom.
Her er antagelsen at et færre antal devices er resultat af mindre omkring færden.
De personer der meldte sad/lonely/depressed havde færre totale interaktioner for samme interval, samt færre Bluetooth og WIFI enheder opdaget.
De personer der meldte stress havde ligeså et fald i samtlige mobile målinger, desuden faldt antallet af \textit{unikke} sociale interaktioner.

\paragraph{Kritik af metode}
Der bruges et meget simpelt spørgeskema, med seks spørgsmål (oversat):
\begin{enumerate}
\item Har du ondt i halsen eller hoste?
\item Løber din næse, er den stoppet eller nyser du?
\item Har du feber?
\item Har du kastet op, kvalme eller diarré?
\item Har du følt dig ked af det, ensom eller deprimeret på det seneste?
\item Har du følt dig stresset på det seneste?
\end{enumerate}

De spørgsmål der er relaterede til psykisk helbred, er meget overordnede, og giver derfor kun et groft billede.

Spørgeskemaet blev udsendt hver morgen kl. 6, og på trods af at deltagerne blev betalt \$ 1 for hvert udfyldt spørgeskema blev kun 63 \% udfyldt.
Ifølge \citet{panas} der har udført forsøg med henholdsvis 80 og 123 personer, havde forsøgspersoner mindre affekt om morgenen, mens den steg hen over dagen, og faldt igen om aftenen.
Dette indikerer at forespørgselstidspunkt skal tages med i beregningerne, eller at de skal foretages på flere og mere varierede tidspunkter.

\subsection{Using Social Sensing to Understand the Links Between Sleep, Mood, and Sociability}
I \citet{social_sensing_2} bruges smartphone ligeså til indsamling af data.
Her bruges dog kun bluetooth proximity til at tælle antal sociale interaktioner.
Det vil sige at der kun tælles hver gang en forsøgsperson kommer i nærheden af en anden forsøgsperson med forsøgets applikation installeret.
Til indsamling af øvrig data bruges daglige, ugentlige og månedlige spørgeskemaer omkring forhold, social aktivitet, humør, kost, motion og søvn.
Disse blev udfyldt gennem smartphone eller web-applikation.

Eksperimentet blev udført med 54 deltager over en periode på 7 måneder, hvor alle deltagere var del af et samfund med 400 beboere og alle bopæle havde knytning til universitetet.

De 3 undersøgte faktorer; søvn, humør og social aktivitet blev alle simplificeret.
Søvn blev reduceret til 7-værdi skala; 5, 6, 7, 8, 9 timer, samt 2 værdier henholdsvis under og over skalaen.
Humør blev reduceret til en binær værdi; hvor en person for en given dag enten ville være i godt eller dårligt humør.
Social aktivitet blev delt i to; daglig relativ aktivitet i forhold til den dag med højest social aktivitet, samt en aggregeret social aktivitet for hver måned.

Ud fra denne data var der to interessante resultater.
For det første viste det sig at søvnlængden havde en stor effekt på dagens humør.
For det andet, og måske mere overraskende, viste det sig at desto højere antal sociale interaktioner over en dag, jo bedre sov personen om natten.
Derimod var der ingen mærkbar forskel på antallet af sociale interaktioner for en dag med lav søvn-længde (pånær 5 timer som viser omkring 20 \% færre daglige sociale interaktioner sammenlignet med længere søvn).

\paragraph{Kritik af metode}
Det største problem med den overordnede metode til indsamling af data er i forhold til indsamling af data vedr. social aktivitet.
Ansigt-til-ansigt interaktion er kun én del ud af den totale mængde sociale interaktion, og derudover var den kun detekterbar mellem de personer der besad smartphone med dertilhørende bluetooth proximity app.
Derfor kunne det være at dage med lavere social aktivitet kunne være nærmere knyttet til en mindre lyst til at begive sig udenfor, fremfor lyst til social aktivitet, som så måske kunne detekteres på anden vis.
Et eksempel på social aktivitet der ikke kan detekteres af denne metode er interaktion over social medier.

Et andet problem med denne metode er også at kvaliteten af søvn- og humørdata er lav, så ud over en overfladisk forbindelse kan der ikke siges noget om hvordan humør og social aktivitet forholder sig i løbet af dagen.
Det nævnes kun kort at personer der generelt havde færre aggregerede sociale interaktioner, generelt havde dårligere humør.
