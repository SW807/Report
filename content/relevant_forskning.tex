% Begreb
% Metod
% Metrik
% Hypotese

\section{Relevant Forskning}
For at kunne konstruere en model til at repræsentere sammenhængen mellem social aktivitet og sindstilstand, er det nødvendigt først at fastslå en umiddelbar sammenhæng.
Dette gøres ved at undersøge eksisterende forskning indenfor området, som primært er indenfor psykologi-faget.

Vores mål er at mere præcist fastslå denne sammenhæng via mobil data-indsamling.
Dette går imod den måde hvorpå data typisk indsamles indenfor psykologi, hvorfor vi efterfølgende vil undersøge om der allerede er forskning mere relateret til datalogi, hvor denne indsamling foregår mobilt.

\subsection{Sammenhæng mellem humør og social aktivitet}
Ud fra de definitioner der er givet i forrige afsnit, er det humør som vi er interesseret i, da det dækker over længerevarende ændringer, hvilket er tilfældet ved mani-depressive perioder.
Det virker dog som at de undersøgte artikler ofte bruger humør og affekt i flæng, hvilket nok indikerer at der er en sammenhæng og en fin grænse mellem de to.

For at undersøge den psykologiske forskning og eksperimentering der er foregået op til nu, i forhold til sammenhæng mellem humør og social aktivitet, tages der udgangspunkt i \citet{whelan}.

I artiklen påstås det at det er første gang der udføres et eksperiment til at fastslå sammenhængen mellem humør og social aktivitet.
I indledningen findes en undersøgelse af lignende, eksisterende forskning, her tages fat i artikler fra 1986 til selve artiklens publikation (2012).
De kritiserer nogle af disse tidligere artikler for at komme frem med en tvetydig retning, altså at det ikke er til at vide om humør påvirker social aktivitet eller omvendt.
Derudover er der også meget kritik til tidligere brugte metoder, som fejler i at skelne mellem behagelige og sociale aktiviteter.

Den generelle måde hvorpå disse eksperimenter foregår (gælder selve artiklen, samt artiklens kilder) er over en enkelt sammenhængende tidsperiode (en eller flere dage), induceres et bestemt humør, hvorefter der foretages et interview ud fra en velkendt metode, til at fastslå humør.

\subsubsection{Selve artiklen}
Selve artiklen udfører i alt 3 eksperimenter.
I første eksperiment, med 58 deltagere, bliver 28 situationer udvalgt, hvor hver situation er vurderet fra 1 til 5 for socialitet og positivitet.



\subsection{Brug af mobil data-indsamling}
Der er fundet to eksempler på brugen af mobil data-indsamling til at fastslå humør-ændringer; \citet{social_sensing} og \citet{social_sensing_2}.
