% % % OVERORDNET
% Begreb
% Metode
% Metrik
% Hypotese

% % % ARTIKEL (Mood -> Sociability)
% Hypotese: Positivt humør fører til sociale aktiviteter
% Tidligere: sociale aktiviteter -> affekt
% Metode: Inducering af humør, vurdering af humør, vurdering af fundne aktiviteter

\section{Relevant Forskning}
For at kunne konstruere en model til at repræsentere sammenhængen mellem social aktivitet og sindstilstand, er det nødvendigt først at fastslå en umiddelbar sammenhæng.
Dette gøres ved at undersøge eksisterende forskning indenfor området, som primært er indenfor psykologi-faget.

Vores mål er at mere præcist fastslå denne sammenhæng via mobil data-indsamling.
Dette går imod den måde hvorpå data typisk indsamles indenfor psykologi, hvorfor vi efterfølgende vil undersøge om der allerede er forskning mere relateret til datalogi, hvor denne indsamling foregår mobilt.

\subsection{Sammenhæng mellem humør og social aktivitet}
Ud fra de definitioner der er givet i forrige afsnit, er det humør som vi er interesseret i, da det dækker over længerevarende ændringer, hvilket er tilfældet ved mani-depressive perioder.

For at undersøge den psykologiske forskning og eksperimentering der er foregået op til nu, i forhold til sammenhæng mellem humør og social aktivitet, tages der udgangspunkt i \citet{whelan}.

I denne artikel fastslås det at der er meget forskning der dokumenterer en 2-vejs forbindelse mellem affekt og social aktivitet\footnote{Engelsk: \textit{sociability}}.
Dog har der været begrænset med eksperimentering til at fastslå at godt humør eller positiv affekt fører til mere social aktivitet.
Ud fra en undersøgelse af lignende artikler i den eksisterende forskning, her tages fat i artikler fra 1986 til selve artiklens publikation (2012), kritiseres den relaterede forskning.
De kritiserer nogle af disse tidligere artikler for at komme frem med en tvetydig retning, altså at det ikke er til at vide om humør påvirker social aktivitet eller omvendt.
Derudover er der også kritik til de tidligere brugte metoder, som fejler i at skelne mellem behagelige og sociale aktiviteter.




\subsection{Brug af mobil data-indsamling}
Der er fundet to eksempler på brugen af mobil data-indsamling til at fastslå humør-ændringer; \citet{social_sensing} og \citet{social_sensing_2}.
