% % % OVERORDNET
% Begreb
% Metode
% Metrik
% Hypotese

% % % ARTIKEL (Mood -> Sociability)
% Hypotese: Positivt humør fører til sociale aktiviteter
% Tidligere: sociale aktiviteter -> affekt
% Metode: Inducering af humør, vurdering af humør, vurdering af fundne aktiviteter

\section{Relevant Forskning}
Ved at undersøge den nuværende forskning, vil vi først og fremmest fastslå at der er en forbindelse mellem humør og social aktivitet.
Disse vil findes indenfor psykologi-faget.
Vi har en formodning om at måden hvorpå dataindsamling og eksperimentering foregår indenfor dette fag, er ved interview og spørgeskemaer.
Dette mener vi at ville kunne gøres bedre med mere objektive og tidssvarende dataindsamling, som fx. ved løbende forespørgsler via smartphone.
Derfor vil vi til sidst undersøge relevant forskning indenfor dataindsamling på smartphone, specifikt ift. sindstilstand.
Her vil vi også se på allerede eksisterende, indenfor psykologi, spørgeskemaer, for at skabe et grundlag til hvilken slags forespørgsler, samt tidspunkter og frekvens for disse, er relevant.

\subsection{Sammenhæng mellem humør og social aktivitet}
Ud fra de definitioner der er givet i forrige afsnit, er det humør som vi er interesseret i, da det dækker over længerevarende ændringer, hvilket er tilfældet ved mani-depressive perioder.

For at undersøge den psykologiske forskning og eksperimentering der er foregået op til nu, i forhold til sammenhæng mellem humør og social aktivitet, tages der udgangspunkt i \citet{whelan}.

Det fastslås at der er meget forskning der dokumenterer en 2-vejs forbindelse mellem affekt og social aktivitet\footnote{Engelsk: \textit{sociability}}.
Dog har der været begrænset med eksperimentering til at fastslå at godt humør eller positiv affekt fører til mere social aktivitet, hvilket er hypotesen for eksperimentet i artiklen.
Ud fra en undersøgelse af lignende artikler i den eksisterende forskning, her tages fat i artikler fra 1986 til selve artiklens publikation (2012), kritiseres den relaterede forskning.
De kritiserer nogle af disse tidligere artikler for at komme frem med en tvetydig retning, altså at det ikke vides om humør påvirker social aktivitet eller omvendt.
Derudover er der også kritik til de tidligere brugte metoder, som fejler i at skelne mellem behagelige og sociale aktiviteter.

\paragraph{Selve eksperimentet} er delt op i 3 dele.
Første del består i at fastslå aktiviteter med varierende grad af socialitet og behagelighed.
Dette gøres online ud fra en gruppe af 58 studerende, hvor en liste på 82 aktiviteter vurderes på behagelighed og socialitet på en skala fra 1 til 5.
Dette ender ud i en liste af 28 aktiviteter, med en jævn fordeling af behagelighed.

De 2 næste dele er opbygget på samme måde, hvor henholdsvis 138 og 104 psykologi-studerende først ser et video-klip af positiv, neutral eller negativ natur.
Herefter, for at fastslå det nuværende humør, vurderes en række ord, baseret på PANAS (Positive Affect Negative Affect Schedule) \cite{panas}, mellem 1 og 7 for i hvor høj grad de opleves i øjeblikket.
Sidste trin er en evaluering af de 28 tidligere fastslåede aktiviteter; hvor de alle på en skala fra 1 til 7 vurderes hvor meget test-personen har lyst til at udføre aktiviteten i øjeblikket.

Det konkluderes ud fra eksperimentet at positivt humør fører til højere lyst til at udføre mere sociale aktiviteter.
Derudover noteres det også at der var en bemærkelsesværdig tendens for de personer der fik induceret negativt humør, havde en højere lyst til ikke-sociale aktiviteter.

\subsection{Brug af mobil data-indsamling}
Der er fundet to eksempler på brugen af mobil data-indsamling til at fastslå humør-ændringer; \citet{social_sensing} og \citet{social_sensing_2}.

\subsection{Måling af sindstilstand}
En anerkendt liste af begreber relateret til positiv og negativ affekt, som bruges til at vurdere sindstilstand, er \citet{panas}.

