I dette kapitel vil der, med udgangspunkt i forrige kapitel, gennemgås de krav en løsning vil skulle overholde for at kalde den en succes.
Dette er krav til en fleksibel, dækkende og præcis dataindsamling, samt mulighed for at kunne tilpasses diverse former for indsamling (diverse skalaer).

\section{Typer af Spørgsmål}
Her tages udgangspunkt i de tre undersøgte skemaer; PANAS, Stemningsregistrering og HAM-D (se \cref{maaling_af_stemningsleje}).
\mikael{Motivation; hvorfor tages udgangspunkt i disse?}
\mikael{Subsubsections i stedet for paragraphs}

\paragraph{PANAS} er simpelt opbygget, da det er den samme sætning der er opbygget omkring de 20 begreber; \textit{I hvor høj grad har du oplevet [begreb] indenfor [tidsperiode]?}.
Svarmuligheden for alle disse spørgsmål er også den samme, en Likert-skala med værdien 1-5, samt følgende tekster knyttet til værdierne (oversat);
\begin{enumerate}
\item Ganske lidt eller overhovedet ikke
\item Lidt
\item Moderat
\item Ganske meget
\item Ekstremt
\end{enumerate}

Ud fra dette skal det være muligt at forme et simpelt spørgsmål med en Likert-skala værdi som svar.\bruno{Kilde på likert-skala}

\paragraph{Stemningsregistrering} er en smule mere kompleks, da den indeholder 5 forskellige spørgsmål; stemningsleje, angst, antal timers søvn, medicin, notater og vægt, med 4 forskellige typer af svarmuligheder.

\subparagraph{Stemningsleje} besvares med en Likert-skala, hvor værdierne i stedet for tal er 7 farver gående fra mørk rød (manisk) til mørk blå (deprimeret), med gul som normalt stemmeleje.

\subparagraph{Angst} angives med en værdi mellem 0 og 3, altså en Likert-skala der starter på 0 og slutter på 3.

\subparagraph{Antal timers søvn} er en talværdi, antageligt gående fra 0 til 24, da stemningsregistrering udfyldes hver dag.

\subparagraph{Medicin} er en mulighed for at angive hvilket, og på hvor mange dage, medicin har været indtaget den forestående måned.
Dette udfyldes derved kun sidst på måneden.

\subparagraph{Notater og vægt} er en mulighed for at notere vigtige hændelser som fri tekst.
Her vil kvinder også skulle angive dage med menstruation.
Desuden skal vægten angives mindst én gang om måneden.

\paragraph{HAM-D} er ligesom PANAS mere simpel.
Den består af 17 faste spørgsmål, hvor hvert spørgsmål har 3-5 faste svarmuligheder.
For hvert spørgsmål vælges ét svar.
\mikael{Flyt op under PANAS, kom med eksempel på spørgsmål og svarmuligheder}

\subsection{Definition af Typer}
\mikael{Her skal uddybes jf. Ivans kommentarer, samt knyttes bedre til forrige sektion}
Her vil de fundne typer blive defineret.
Ens for alle er at de kan formuleres som spørgsmål, det er til gengæld svarmulighederne der varierer.
Ud fra ovenstående er der 4 typer af svarmuligheder;

\begin{enumerate}
\item Angivelse af værdi på Likert-skala \bruno{Kilde på likert-skala}
\begin{itemize}
\item Som også skal kunne angives med farver i stedet for tal
\end{itemize}
\item Fri tekst
\item Markering af dage på månedskalender
\item Valg mellem tekstuelle predefinerede svarmuligheder
\end{enumerate}

\section{Tidspunkter for Spørgsmål}
Ud fra den forrige sektion, er der lagt op til at spørgsmål skal stilles på forskellige tidspunkter af måneden.\mikael{Mere generel tid}

Samtidig, ud fra undersøgelsen af den relevante forskning, er det også relevant at stille spørgsmålene på det rigtige tidspunkt for døgnet.
Blandt andet fandt de ud af i \citet{PANAS} at der var en generel tendens til at stemningslejet startede relativt lavt om morgenen, for at stige som dagen forløb, for igen at falde hen imod aftenen.

\subsection{Tid på Kalendermåneden}
Som angivet i forrige sektion, vil der for Stemningslejeregistrering være behov for at stille daglige spørgsmål, men også spørgsmål der kun skal besvares en enkelt gang pr. måneden.

\subsection{Tid på Døgnet}
Som benævnt ovenover kan stemningslejet variere i løbet af dagen.
Det er derfor vigtigt at enten stille spørgsmålene på det rigtige tidspunkt, eller tage højde for disse variationer i vægtningen af svaret afhængigt af spørge-tidspunktet.

For PANAS, grundet dets generelhed, kan anvendes på alle tidspunkter af døgnet og med tanke på både kortere og længere tilbageblik.

\section{Yderlige Overvejelser}
Her vil være en diskussion ift. andre overvejelser der kan gøres i forbindelse med de spørgsmål der stilles til brugeren.

\subsection{Reduktion og Variation}
Antageligvis vil det hurtigt blive trættende at skulle svare på en lang række spørgsmål, især hvis dette skal gøres én eller flere gange dagligt.
Dette kunne lede til rutineprægede besvarelser, eller blot for hurtige og uinteresserede besvarelser.
Dette kunne for eksempel være for PANAS at hver dag, eller måske flere gange dagligt, at skulle svare på alle 20 spørgsmål.
Hvis disse spørgsmål kom alle 20 ad gangen, og i samme rækkefølge, ville det være muligt at huske (måske endda ubevist) en rutine i udfyldningen af svarene.

Dog er det muligt, eftersom spørgsmålene for PANAS ikke er tæt knyttet, at udvælge nogle af disse ad gangen, og sprede dem ud over dagen(e).
Dette ville gøre det muligt at variere tidspunktet det enkelte spørgsmål stilles på, samt hvor ofte.

\subsection{Adaption og Refleksion}
Hvis vi antager at vi har en lang række spørgsmål fra de forskellige skemaer, vil dette være en betydelig mængde med varierende grad.
For eksempel giver de enkelte PANAS spørgsmål ikke et godt billede af stemning i sig selv, men kun kombinationen af dem gør dette.
Hvorimod for Stemningsregistrering og HAM-D gælder det modsatte, da de enkelte spørgsmål er længere og mere dækkende.

Ved at bruge nogle korte og overordnede spørgsmål som udgangspunkt, kan der efter behov og mistanke om forværring ændres til længere og mere præcise spørgsmål.
På denne måde belastes brugeren som udgangspunkt ikke, men først når der kan detekteres en begyndende forværring vil der spørges mere ind til dette.

Yderligere kan det også være godt at tvinge brugeren til at reflektere over sin egen tilstand.
Ved begyndende forværring kunne brugeren gøres opmærksom på dette og derved gives mulighed for at overveje om der er hold i dette.
Dette kunne kombineres med at vise tendenserne for de tidligere besvarelser.
\mikael{Masser af Ivan kommentar her}
