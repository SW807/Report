I dette kapitel vil der, med udgangspunkt i forrige kapitel, gennemgås de krav en løsning vil skulle overholde for at kalde den en succes.
Dette er krav til en fleksibel, dækkende og præcis dataindsamling, samt mulighed for at kunne tilpasses diverse former for indsamling (diverse skalaer).

\section{Typer af Spørgsmål}
Her tages udgangspunkt i de tre undersøgte skemaer; PANAS, Stemningsregistrering og HAM-D (se \cref{maaling_af_stemningsleje}).

\paragraph{PANAS} er simpelt opbygget, da det er den samme sætning der er opbygget omkring de 20 begreber; \textit{I hvor høj grad har du oplevet [begreb] indenfor [tidsperiode]?}.
Svarmuligheden for alle disse spørgsmål er også den samme, en Likert-skala med værdien 1-5, samt følgende tekster knyttet til værdierne (oversat);
\begin{enumerate}
\item Ganske lidt eller overhovedet ikke
\item Lidt
\item Moderat
\item Ganske meget
\item Ekstremt
\end{enumerate}

Ud fra dette skal det være muligt at forme et simpelt spørgsmål med en Likert-skala værdi som svar.

\paragraph{Stemningsregistrering} er en smule mere kompleks, da den indeholder 5 forskellige spørgsmål; stemningsleje, angst, antal timers søvn, medicin, notater og vægt, med 4 forskellige typer af svarmuligheder.

\subparagraph{Stemningsleje} besvares med en Likert-skala, hvor værdierne i stedet for tal er 7 farver gående fra mørk rød (manisk) til mørk blå (deprimeret), med gul som normalt stemmeleje.

\subparagraph{Angst} angives med en værdi mellem 0 og 3, altså en Likert-skala der starter på 0 og slutter på 3.

\subparagraph{Antal timers søvn} er en talværdi, antageligt gående fra 0 til 24, da stemningsregistrering udfyldes hver dag.

\subparagraph{Medicin} er en mulighed for at angive hvilket, og på hvor mange dage, medicin har været indtaget den forestående måned.
Dette udfyldes derved kun sidst på måneden.

\subparagraph{Notater og vægt} er en mulighed for at notere vigtige hændelser som fri tekst.
Her vil kvinder også skulle angive dage med menstruation.
Desuden skal vægten angives mindst én gang om måneden.

\paragraph{HAM-D} er ligesom PANAS mere simpel.
Den består af 17 faste spørgsmål, hvor hvert spørgsmål har 3-5 faste svarmuligheder.
For hvert spørgsmål vælges ét svar.

\subsection{Definition af Typer}
Her vil de fundne typer blive defineret.
Ens for alle er at de kan formuleres som spørgsmål, det er til gengæld svarmulighederne der varierer.
Ud fra ovenstående er der 4 typer af svarmuligheder;

\begin{enumerate}
\item Angivelse af værdi på Likert-skala
\begin{itemize}
\item Som også skal kunne angives med farver i stedet for tal
\end{itemize}
\item Fri tekst
\item Markering af dage på månedskalender
\item Valg mellem tekstuelle predefinerede svarmuligheder
\end{enumerate}
