I dette kapitel vil der, med udgangspunkt i forrige kapitel, gennemgås de krav en løsning vil skulle overholde for at kalde den en succes.
Dette er krav til en fleksibel, dækkende og præcis dataindsamling, samt mulighed for at kunne tilpasses diverse former for indsamling (diverse skalaer).

\section{Typer af Spørgsmål}
For at undersøge hvilke typer af spørgsmål der er nødvendige, overvejes det hvilke spørgsmål metoderne beskrevet i \cref{maaling_af_stemningsleje} benytter sig af.
Dette afsnit vil derfor tage udgangspunkt i de tre undersøgte skemaer; PANAS, Stemningsregistrering og HAM-D.

\subsection{PANAS} 
PANAS er simpelt opbygget, da det er den samme sætning der er opbygget omkring de 20 begreber; \textit{I hvor høj grad har du oplevet [begreb] indenfor [tidsperiode]?}.
Svarmuligheden for alle disse spørgsmål er også den samme, en Likert-skala\cite{likert} med værdien 1-5, samt følgende tekster knyttet til værdierne (oversat);
\begin{enumerate}
\item Ganske lidt eller overhovedet ikke
\item Lidt
\item Moderat
\item Ganske meget
\item Ekstremt
\end{enumerate}

Ud fra dette skal det være muligt at forme et simpelt spørgsmål med en Likert-skala værdi som svar.

\subsection{HAM-D} 
HAM-D er ligesom PANAS simpelt opbygget.
Den består af 17 faste faktorer, hvor hver faktor har 3-5 faste sætninger hvor patienten skal vælge den mulighed der beskriver hans opfattelse af faktoren bedst muligt.
For hver faktor vælges ét svar.

Et eksempel på en faktor fra HAM-D \cite{ham_d}:\\

\textbf{Arbejde og aktiviteter}
\begin{itemize}
\item Ingen problemer
\item Tanker eller følelser om inhabilitet relateret til arbejde og aktiviteter
\item Mistet interesse i arbejde og aktiviteter 
\end{itemize}

Det er altså nødvendigt at kunne lave et spørgsmål der indeholder en titel og et antal svarmuligheder for at kunne vise HAM-D baserede spørgsmål.

\subsection{Stemningsregistrering} 
Stemningsregistrering er en smule mere kompleks, da den indeholder 5 forskellige spørgsmål; stemningsleje, angst, antal timers søvn, medicin, notater og vægt, med 4 forskellige typer af svarmuligheder.

\paragraph{Stemningsleje} besvares med en Likert-skala, hvor værdierne i stedet for tal er 7 farver gående fra mørk rød (manisk) til mørk blå (deprimeret), med gul som normalt stemmeleje.

\paragraph{Angst} angives med en værdi mellem 0 og 3, altså en Likert-skala der starter på 0 og slutter på 3.

\paragraph{Antal timers søvn} er en talværdi, antageligt gående fra 0 til 24, da stemningsregistrering udfyldes hver dag.

\paragraph{Medicin} er en mulighed for at angive hvilket, og på hvor mange dage, medicin har været indtaget den forestående måned.
Dette udfyldes derved kun sidst på måneden.

\paragraph{Notater og vægt} er en mulighed for at notere vigtige hændelser som fri tekst.
Her vil kvinder også skulle angive dage med menstruation.
Desuden skal vægten angives mindst én gang om måneden.

Til stemningsregistrering er der behov for flere typer spørgsmål.
For at kunne stille \emph{stemningsleje} spørgsmål skal man kunne opstille en Likert skala med forskellige farver på svarmulighederne.
For at kunne spørge til patientens \emph{angst} er det nødvendigt at kunne opstille en talskala fra 0 til 3.
\emph{Antal timers søvn}, \emph{medicin} og \emph{Notater og vægt} kræver at patienten selv kan skrive sine iagttagelser.

\subsection{Definition af Typer}\label{krav::typer}
Efter at have undersøgt metoder der bruges i psykologien kan de resulterende typer af spørgsmål defineres.
Ens for alle er at de kan formuleres som spørgsmål, det er til gengæld svarmulighederne der varierer.
Ud fra ovenstående er de resulterende typer af svarmuligheder:

\begin{enumerate}
\item Angivelse af værdi på Likert-skala
\begin{itemize}
\item Skal kunne tilægges farver
\end{itemize}
\item Fri tekst
\item Valg mellem tekstuelle predefinerede svarmuligheder
\end{enumerate}

\section{Tidspunkter for Spørgsmål}
Ud fra den forrige sektion, er der lagt op til at spørgsmål skal stilles på forskellige tidspunkter.

Det kan både være forskellige tidspunkter på måneden eller forskellige tidspunkter på dagen.
Blandt andet viste undersøgelsen i \citet{panas} om PANAS, at der var en generel tendens til at stemningslejet startede relativt lavt om morgenen, for at stige som dagen forløb, for igen at falde hen imod aftenen.

Det følgende vil beskrive hvilke muligheder for indstilling af tid der er nødvendig for at kunne udføre de metoder der blev undersøgt i \cref{maaling_af_stemningsleje}.

\subsection{Tid på Kalendermåneden}
Som angivet i forrige sektion, vil der for Stemningslejeregistrering være behov for at stille daglige spørgsmål, men også spørgsmål der kun skal besvares en enkelt gang pr. måneden.

\subsection{Tid på Døgnet}
Som benævnt ovenover kan stemningslejet variere i løbet af dagen.
Det er derfor vigtigt at enten stille spørgsmålene på det rigtige tidspunkt, eller tage højde for disse variationer i vægtningen af svaret afhængigt af spørge-tidspunktet.

\section{Yderlige Overvejelser}
Her vil være en diskussion ift. andre overvejelser der kan gøres i forbindelse med de spørgsmål der stilles til brugeren.
\stefan{for generel overskrift hvis vi ikke skriver andet}

\subsection{Reduktion og Variation}
Antageligvis vil det hurtigt blive trættende at skulle svare på en lang række spørgsmål, især hvis dette skal gøres én eller flere gange dagligt.
Dette kunne lede til rutineprægede besvarelser, eller blot for hurtige og uinteresserede besvarelser.
Dette kunne for eksempel være for PANAS at hver dag, eller måske flere gange dagligt, at skulle svare på alle 20 spørgsmål.
Hvis disse spørgsmål kom alle 20 ad gangen, og i samme rækkefølge, ville det være muligt at huske (måske endda ubevist) en rutine i udfyldningen af svarene.

Det er derfor vigtigt at overveje hvor mange spørgsmål man stiller ad gangen, og altså foretage en \emph{reduktion} af spørgsmålsmængden, så patienten har overskud til at besvare dem alle.

Når spørgsmålene er blevet reduceret vil det være muligt at lave en \emph{variation} over de benyttede spørgsmål.
Her kan der benyttes flere strategier.
Man kan vente et fast interval før man stiller et spørgsmål igen, men man kan også variere intervallet så det bliver mindre forudsigeligt.

\subsection{Adaption og Refleksion}
Hvis vi antager at vi har en lang række spørgsmål fra de forskellige skemaer, vil dette være en betydelig mængde som beskriver patientens tilstand i varierende grad.
For eksempel giver de enkelte PANAS spørgsmål ikke et godt billede af stemning i sig selv, men kombinationen af spørgsmål.
På den anden side gælder det modsatte for Stemningsregistrering og HAM-D, da de enkelte spørgsmål er længere og mere dækkende.

Ved at bruge nogle korte og overordnede spørgsmål som udgangspunkt, kan der efter behov og mistanke om forværring ændres til længere og mere præcise spørgsmål.
På denne måde belastes brugeren som udgangspunkt ikke, men først når der kan detekteres en begyndende forværring vil der spørges mere ind til dette.

Yderligere kan det også være godt at tvinge brugeren til at reflektere over sin egen tilstand.
Ved begyndende forværring kunne brugeren gøres opmærksom på dette og derved gives mulighed for at overveje om der er hold i dette.
Dette kunne kombineres med at vise tendenserne for de tidligere besvarelser.
\mikael{Masser af Ivan kommentar her}
\stefan{stefan skriv, dude}
