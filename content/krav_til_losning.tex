\chapter{Behov til løsning}\label{krav}
I dette kapitel vil der, med udgangspunkt i forrige kapitel, gennemgås de behov der er opstået til en mulig løsning.
Behovenes formål er at konstruere en fleksibel, dækkende og præcis dataindsamling på baggrund af de eksisterende psykologiske metoder.

\section{Typer af spørgsmål}
For at undersøge hvilke typer af spørgsmål der er nødvendige, overvejes det hvilke spørgsmål som metoderne beskrevet i \cref{maaling_af_stemningsleje} benytter sig af.
Dette afsnit vil derfor tage udgangspunkt i de tre undersøgte skemaer: PANAS, Stemningsregistrering og HAM-D.

\subsection{PANAS} 
PANAS er simpelt opbygget, da det er den samme sætning der er opbygget omkring de 20 begreber; \textit{I hvor høj grad har du oplevet [begreb] indenfor [tidsperiode]?}.
Svarmuligheden for alle disse spørgsmål er også den samme, en Likert-skala\cite{likert} med værdien 1-5, samt følgende tekster knyttet til værdierne (oversat);
\begin{enumerate}
\item Ganske lidt eller overhovedet ikke
\item Lidt
\item Moderat
\item Ganske meget
\item Ekstremt
\end{enumerate}

Ud fra dette skal det være muligt at forme et simpelt spørgsmål med en Likert-skala værdi som svar.

\subsection{HAM-D} 
HAM-D er ligesom PANAS simpelt opbygget.
Den består af 17 faste faktorer, hvor hver faktor har 3-5 faste sætninger hvor patienten skal vælge den mulighed der beskriver hans opfattelse af faktoren bedst muligt.
For hver faktor vælges ét svar.

\noindent
Et eksempel på en faktor fra HAM-D \cite{ham_d}:\\

\textbf{Arbejde og aktiviteter}
\begin{itemize}
\item Ingen problemer
\item Tanker eller følelser om inhabilitet relateret til arbejde og aktiviteter
\item Mistet interesse i arbejde og aktiviteter 
\end{itemize}

\noindent
Det er altså nødvendigt at kunne lave et spørgsmål der indeholder en titel og et antal svarmuligheder for at kunne vise HAM-D baserede spørgsmål.

\subsection{Stemningsregistrering}\label{krav:stemningsregistrering}
Stemningsregistrering er en smule mere kompleks, da den indeholder 5 forskellige spørgsmål; stemningsleje, angst, antal timers søvn, medicin, notater og vægt, med 4 i alt forskellige typer af svarmuligheder.

\begin{description}
\item[Stemningsleje] besvares med valgmuligheder fra 'svær depression' til 'svær mani' der er indikeret med farver gående fra mørk rød (svær mani) til mørk blå (svær depression), med gul som normalt stemningsleje (se evt. \cref{figure::stemningsregistrerings_skema}).

\item[Angst] angives med en værdi mellem 0 og 3, altså en Likert-skala der starter på 0 og slutter på 3.

\item[Antal timers søvn] er en talværdi, antageligt gående fra 0 til 24, da stemningsregistrering udfyldes hver dag.

\item[Medicin] er en mulighed for at angive hvilket, og på hvor mange dage, medicin har været indtaget den forestående måned.
Dette udfyldes derved kun sidst på måneden.

\item[Notater og vægt] er en mulighed for at notere vigtige hændelser som fritekst.
Her vil kvinder også skulle angive dage med menstruation.
Desuden skal vægten angives mindst én gang om måneden.
\end{description}

Til stemningsregistrering er der behov for flere typer spørgsmål.
For at kunne stille \emph{stemningsleje} spørgsmål skal man kunne opstille en Likert skala med forskellige farver på svarmulighederne.
For at kunne spørge til patientens \emph{angst} er det nødvendigt at kunne opstille en talskala fra 0 til 3.
\emph{Antal timers søvn}, \emph{medicin} og \emph{Notater og vægt} kræver at patienten selv kan skrive sine iagttagelser.

\subsection{Definition af typer}\label{krav::typer}
Efter at have undersøgt metoder der bruges i psykologien kan de resulterende typer af spørgsmål defineres.
Ens for alle er at de kan formuleres som spørgsmål, det er til gengæld svarmulighederne der varierer.
Ud fra ovenstående er de resulterende typer af svarmuligheder:

\begin{enumerate}
\item Angivelse af værdi på Likert-skala
\item Fri tekst
\item Valg mellem tekstuelle predefinerede svarmuligheder
\begin{itemize}
	\item Skal kunne tillægges farver
\end{itemize}
\end{enumerate}

\section{Tidspunkter for spørgsmål}\label{tidspnkforspg}
Ud fra den forrige sektion er der lagt op til, at spørgsmål skal stilles på forskellige tidspunkter.
Det kan både være forskellige tidspunkter på måneden og forskellige tidspunkter på dagen.
Blandt andet viste undersøgelsen i \citet{panas} om PANAS, at der var en generel tendens til at stemningslejet startede relativt lavt om morgenen, steg som dagen forløb for igen at falde hen imod aftenen.

Det er projektgruppens ønske at etablere en løsning for hvilken det er muligt at præcisere hvornår patienten skal stilles bestemte spørgsmål.
Det følgende vil derfor beskrive hvilke muligheder for indstilling af tid der er nødvendige for at kunne udføre de metoder der blev undersøgt i \cref{maaling_af_stemningsleje}.

\paragraph{Tid på kalendermåneden}
Som angivet i \cref{krav:stemningsregistrering}, vil der for stemningslejeregistrering være behov for at stille daglige spørgsmål, men også spørgsmål der kun skal besvares en enkelt gang om måneden.

\paragraph{Tid på døgnet}
Som benævnt ovenover kan stemningslejet variere i løbet af dagen.
Det er derfor vigtigt at enten stille spørgsmålene på det rigtige tidspunkt, eller tage højde for disse variationer i vægtningen af svaret afhængigt af spørge-tidspunktet.
Dette lægger op til følgende åbne problemstilling: \textit{Hvornår skal spørgsmål stilles for at få det bedste billede af patientes stemningsleje?}
En af problemstillingerne herved er forståelsen af ''\textit{det rigtige tidspunkt}''.
Ved at udvælge tidspunktet for hvornår brugeren spørges påvirkes resultatet af besvarelserne.
Dermed skal analyse af besvarelser ses i kontekst af hvornår besvarelsen er sket.
Alternativt skal spørgsmål, som nævnt herover, stilles på et tidspunkt således at tidspunktet har minimal effekt på besvarelsen.
En mulighed for at opnå dette er at lade brugeren besvare samme spørgsmål på samme tidspunkt, således at besvarelser kan vurderes i relation til hinanden med ensartet påvirkning fra tidspunktet.

