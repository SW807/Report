\chapter{Dataindsamling}
Der findes adskillige muligheder for logning på Android-baserede mobiltelefoner \citefaelles{Sektion 3.3 Mobil logning}.
Disse inkluderer blandt andet opkald, applikationsbrug og SMS/MMS.
Derudover kan GPS \citefaelles{Sektion 3.1 Sensorer} også anvendes som indikation på aktivitet, herunder social aktivitet.

Alle henvisninger til patienters adfærd kan antages som værende udenfor maniske eller depressive perioder.
Dette gøres da det er vigtigt at opdage ændring i adfærden så tidligt som muligt, da lige så snart en episode er indtruffet vil applikationen ikke længere kunne assistere \citefaelles{Sektion 1.5 Fokusgruppeinterview}.
Det er derfor nødvendigt at etablere en norm for den enkelte patient, således at ændringer i netop dennes adfærd kan detekteres.

\subsection{GPS}
Via smartphone, eller wearable som understøtter dette, er det muligt at forespørge på patientens nuværende lokation.
Dette resulterer, som minimum \mikkel{Som minimum? Hvad er der mere at hente?}, i en længdegrad, breddegrad og præcision.
Derudover kan flere forespørgsler kombineres til at sige noget om retning, hastighed og acceleration.

Nogle patienter vil deltage i sociale aktiviteter, såsom sport eller anden klub-aktivitet, fest og generel social aktivitet med familie eller venner.
Så ville man over tid kunne identificere nøgle-lokationer (fx. hjem, skole/arbejde, sportsklub, ven), da de ville være tilbagevendende.
Evt. kunne man få bruger-input til præcist at identificere nogle af disse lokationer.
Når først disse lokationer er bestemt, vil man kunne danne en model over den enkelte patients aktivitets-mønster.

Ved at skelne mellem sociale (venner, familie, sportsklub) og ikke-sociale lokationer (ens hjem, muligvis andre), ville man kunne registrere ændringer i den sociale aktivitet.
Ud fra den viden der er dannet i fællesrapporten, vil der for patienter på vej ind i en depression være et fald i social aktivitet.
For patienter på vej ind i en mani vil der omvendt være en stigning.

\paragraph{Begrænsninger}
Ved GPS-lokation alene, er det ikke direkte muligt at sige om en aktivitet er social eller ej.
Der vil være situationer hvor man er i ens hjem en hel weekend, men muligvis har besøg af venner eller familie.
Omvendt kunne man også tage til en familiebegivenhed, hvor man ikke nødvendigvis snakker eller på anden måde socialiserer med nogen.

\subsection{Opkald}
På Android-smartphones er det muligt at få adgang til opkalds-historikken.
Her kan man se indgående, udgående og mistede opkald\footnote{Et mistet opkald er et indgående opkald som ikke besvares}, samt kontaktoplysninger og varighed af det enkelte opkald.

Denne meta-information om de opkald patienten foretager vil kunne danne grundlag for patientens normale adfærd.
Ændringer heri kan være en indikation på en ændring i sindstilstand.
Ved at tælle antallet af opkald, ville man kunne observere en ændring ved henholdsvis begyndende depression eller mani.
Ved depression antages det at antallet af opkald, især udgående, vil falde.
Tilsvarende kan det antages at patienten har flere mistede opkald end det er normalt for patienten.
Det antages også at længden af opkaldene vil være kortere end normalt.
Som benævnt i fokusgruppe-interviewet, vil ens netværk også mindskes, hvilket vil kunne ses på modtager/afsender på opkaldene, hvor antallet af forskellige kontaktpersoner vil forventes at falde.
Derimod vil man ved mani forvente en stigning i opkald, især udgående.
Der vil også være en stigning i antal forskellige udgående kontakter.

\paragraph{Begrænsninger}
Eftersom der kun arbejdes med meta-information om opkaldene, er det ikke direkte muligt at sige noget om kvaliteten af opkaldene.
Det vil sige at der ikke kan siges noget om hvor meget og hvad der bliver sagt i den enkelte samtale.

For mange personer \mikkel{Har vi en kilde på 'mange'?} vil der være begrænset brug af telefon-opkald, da en del af kommunikationen vil foregå via andre platforme som fx Facebook, Whatsapp eller Skype.

\subsection{SMS/MMS beskeder}
Ligesom opkalds-historik, er der i Android mulighed for adgang til SMS/MMS beskeder.
Her kan man få adgangs til lister af beskeder\footnote{Indbakke, udbakke, sendte beskeder etc.}, med for hver besked afsender/modtager, indhold (og derudaf længden) samt afsendt/modtaget tidspunkt.

Observation af beskeder vil fungere i høj grad på samme måde som opkald.
Det betyder at der heller ikke med beskeder fokuseres på kommunikationens indhold men i stedet på dens volumen.

\paragraph{Begrænsninger}
Da indsamlingen af besked-information i høj grad ligner indsamlingen af opkalds-information gør de samme begrænsninger sig også gældende her.

\subsection{Applikations-brug}
I nyere versioner af Android \footnote{Android 5.0 Lollipop} er det muligt at få adgang til brugs-statistik for alle installerede applikationer.
Her er det muligt at forespørge på forskellige tidsintervaller, hvilket resulterer i en liste af applikationer brugt i tidsintervallet.
For hver applikation vil der være informationer omkring totalt tidsforbrug i millisekunder, tidligste og seneste brug (indenfor angivet interval).

Med denne information er det muligt at se hvor meget tid der bruges på applikationer med socialt indhold, som for eksempel Facebook, Skype e.l.

\paragraph{Begrænsninger}
Det er ikke muligt at se hvad de forskellige apps har været brugt til, men kun i hvor lang tid.
For eksempel vil man for Facebook-app'en ikke kunne se om der er brugt tid på at skrive til andre, læse nyheder eller spille spil.

\subsection{Andre ikke-udforskede muligheder}

\paragraph{Kalender}

\paragraph{Sociale apps indhold}
