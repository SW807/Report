\section{Dataindsamling}
I \cref{?}\footnote{Fællesrapport, Sektion 3.3 Mobil logning} blev der nævnt adskillige muligheder for logning på mobiltelefon.
Disse var; opkald, applikationsbrug og SMS/MMS.
Derudover blev der i \cref{?}\footnote{Fællesrapport, Sektion 3.1 Sensorer} nævnt GPS, hvilket også ville kunne blive brugt som indikation på aktivitet, herunder social aktivitet.

Eksempler på brugen af disse logninger, i forbindelse med symptomerne (se \cref{?}\footnote{Fællesrapporten, Sektion 1.21 Depression og 1.2.2 Mani}), blev analyseret i \cref{?}\footnote{Fællesrapporten, Sektion 3.4}.
Disse vil blive uddybet her, med eksempel på brug, samt begrænsninger, ved de forskellige logninger.

Ved relatering til patienters adfærd, antages det at disse er udenfor, eller mellem, episode.
Derved antages det at disse fungerer som almindelige mennesker, der går på arbejde eller i skole og deltager i andre aktiviteter (herunder sociale).

\subsection{GPS}
Via smartphone, eller wearable som understøtter dette, er det muligt at forespørge på brugerens nuværende lokation.
Dette resulterer, som minimum, i en længdegrad, breddegrad og præcision.
Derudover kan flere forespørgsler kombineres til at sige noget om retning, hastighed og acceleration.

Nogle patienter vil deltage i sociale aktiviteter, såsom sport eller anden klub-aktivitet, fest og generel social aktivitet med familie eller venner.
Så ville man over tid kunne identificere nøgle-lokationer (fx. hjem, skole/arbejde, sportsklub, ven), da de ville være tilbagevendende.
Evt. kunne man få bruger-input til præcist at identificere nogle af disse lokationer.
Når først disse lokationer er bestemt, vil man kunne danne en model over den enkelte patients aktivitets-mønster.

Ved at skelne mellem sociale (venner, familie, sportsklub) og ikke-sociale lokationer (ens hjem, muligvis andre), ville man kunne registrere ændringer i den sociale aktivitet.
Ud fra den viden der er dannet i fællesrapporten, vil der for patienter på vej ind i en depression være et fald i social aktivitet.
For patienter på vej ind i en mani vil der omvendt være en stigning.

\paragraph{Begrænsninger}
Ved GPS-lokation alene, er det ikke direkte muligt at sige om en aktivitet er social eller ikke.
Der vil være situationer hvor man er i ens hjem en hel weekend, men muligvis har besøg af venner eller familie.
Omvendt kunne man også tage til en familiebegivenhed, hvor man ikke nødvendigvis snakker eller på anden måde socialiserer med nogen.

\section{Opkald}
På Android-smartphones er det muligt at få adgang til opkalds-historikken.
Her kan man se indgående, udgående og missede opkald, samt modtager/afsender og varighed.

Opkalds-historikken alene kan give et billede af kvantiteten for opkald, samt en smule om kvaliteten via opkaldets varighed.
Ved at tælle antallet af opkald, ville man kunne observere en ændring ved henholdsvis begyndende depression eller mani.
Ved depression antages det at antallet af opkald, især udgående, vil falde.
Det antages også at længden af opkaldene vil være kortere end normalt.
Som benævnt i fokusgruppe-interviewet, vil ens netværk også mindskes, hvilket vil kunne ses på modtager/afsender på opkaldene, hvor antal forskellige kontaktpersoner vil falde.
Derimod vil man ved mani forvente en stigning i opkald, især udgående.
Der vil også være en stigning i antal forskellige udgående kontakter.

\paragraph{Begrænsninger}
Da der kun ses på opkalds-historikken, er det ikke direkte muligt at sige noget om kvaliteten af opkaldene.
Det vil sige at der ikke kan siges noget om hvor meget der bliver sagt af patienten, eller af afsender/modtager.

For mange personer vil der være begrænset brug af telefon-opkald, da der også vil foregå en stor del af den samme slags kommunikation via Facebook, Whatsapp, Skype eller lignende.

\section{SMS/MMS beskeder}
Ligesom opkalds-historik, er der indbygget i Android mulighed for adgang til SMS/MMS-historik.
Her kan man få en liste over beskeder, med for hver besked afsender/modtager, indhold (og derudaf længden), afsendt/modtaget tidspunkt og om hvorvidt det er en sendt eller modtaget besked.

Observation af beskeder vil fungere i høj grad på samme måde som opkald.

\paragraph{Begrænsninger}
For mange personer vil der være begrænset brug af SMS/MMS-beskeder, da der også vil foregå en stor del af den samme slags kommunikation via Facebook, Whatsapp, Skype eller lignende.

\section{Applikations-brug}
I nyere versioner af Android er det muligt at få adgang til brugs-statistik for alle installerede applikationer.
Her er det muligt at forespørge på forskellige tidsintervaller, hvilket resulterer i en liste af applikationer brugt i tidsintervallet.
For hver applikation vil der være informationer omkring totalt tidsforbrug i millisekunder, tidligste og seneste brug (indenfor angivet interval).

Med denne information er det muligt at se hvor meget tid der bruges på applikationer med socialt indhold, som for eksempel Facebook, Skype e.l.

\paragraph{Begrænsninger}
Dette er kun muligt for nyere versioner af Android; fra og med API version 21, eller Androd 5.0 Lollipop.
Det vil heller ikke være muligt at se hvad de forskellige apps har været brugt til, men kun i hvor lang tid.
For eksempel vil man for Facebook-app'en ikke kunne se om der er brugt tid på at skrive til andre, læse nyheder eller spillet spil.

\section{Andre ikke-udforskede muligheder}

\paragraph{Kalender}

\paragraph{Sociale apps indhold}
