\section{Dataindsamling}
I \cref{?}\footnote{Fællesrapport, Sektion 3.3 Mobil logning} blev der nævnt adskillige muligheder for logning på mobiltelefon.
Disse var; opkald, applikationsbrug og SMS/MMS.
Derudover blev der i \cref{?}\footnote{Fællesrapport, Sektion 3.1 Sensorer} nævnt GPS, hvilket også ville kunne blive brugt som indikation på aktivitet, herunder social aktivitet.

Eksempler på brugen af disse logninger, i forbindelse med symptomerne (se \cref{?}\footnote{Fællesrapporten, Sektion 1.21 Depression og 1.2.2 Mani}), blev analyseret i \cref{?}\footnote{Fællesrapporten, Sektion 3.4}.
Disse vil blive uddybet her, med fordele og begrænsninger ved de forskellige logninger.

Ved relatering til patienters adfærd, antages det at disse er udenfor, eller mellem, episode.
Derved antages det at disse fungerer som almindelige mennesker, der går på arbejde eller i skole og deltager i andre aktiviteter (herunder sociale).

\subsection{GPS}
Via smartphone, eller wearable som understøtter dette, er det muligt at forespørge på brugerens nuværende lokation.
Dette resulterer, som minimum, i en længdegrad, breddegrad og præcision.
Derudover kan flere forespørgsler kombineres til at sige noget om retning, hastighed og acceleration.

Nogle patienter vil de deltage i sociale aktiviteter, såsom sport eller anden klub-aktivitet, fest og generel social aktivitet med familie eller venner.
Så ville man over tid kunne identificere disse lokationer, da de ville være tilbagevendende.
Evt. kunne man få bruger-input til præcist at identificere nogle af disse lokationer.
Når først disse lokationer er bestemt, vil man kunne danne en model over den enkelte patients aktivitets-mønster.

Ved at skelne mellem sociale og ikke-sociale lokationer (ens hjem, muligvis andre), ville man kunne registrere ændringer i den sociale aktivitet.
Ud fra den viden der er dannet i fællesrapporten, vil der for patienter på vej ind i en depression være et fald i social aktivitet.
For patienter på vej ind i en mani vil der omvendt være en stigning.

\paragraph{Begrænsninger}
Ved GPS-lokation alene, er det ikke direkte muligt at sige om en aktivitet er social eller ikke.
Der vil være situationer hvor man er i ens hjem en hel weekend, men muligvis har besøg af venner eller familie.
Omvendt kunne man også tage til en familiebegivenhed, hvor man ikke nødvendigvis snakker eller på anden måde socialiserer med nogen.


