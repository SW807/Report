\chapter{Opsamling}
I dette projekt er der blevet gjort klar til indsamling af forstyrrende\footnote{For definition se \cref{begreber::forstyrrende}} og ikke-forstyrrende\footnote{For definition se \cref{begreber::ikke_forstyrrende}} data. 

Som nævnt i \cref{problem} blev disse to indsamlingsmetoder udarbejdet for at udføre et eller flere eksperimenter, som ser på om der er en sammenhæng mellem ikke-forstyrrende indsamlet data og forstyrrende indsamlet data.

\paragraph{Ikke-forstyrrende}
I \cref{ikke_forstyrrende} er de mulige ikke-forstyrrende datakilder analyseret, beskrevet i \cref{datasamling}.
Desuden er to udvalgte blevet implementeret, opkaldsinformation og SMS/MMS information.
Dette er blevet beskrevet i \cref{implementerede_moduler}.
De to moduler gør det muligt at indsamle ikke-forstyrrende data fra patienten som vil udgøre den ene del af eksperimentets data.

\paragraph{Forstyrrende}
I \cref{forstyrrende} er der blevet kigget på forstyrrende indsamling af data.
Der er i \cref{maaling_af_stemningsleje} fundet frem til tre \textit{spørgeskemaer} som har lagt op til nogle behov, beskrevet i \cref{krav}.
Disse behov har været grundstenen for det implementerede modul 'survey', beskrevet i \cref{survey}.
Modulet gør det muligt at indsamle forstyrrende data fra patienten omkring hans stemningsleje.
Dette vil udgøre den anden del af det indsamlede data der skal sammenlignes i eksperimentet.

\paragraph{Indsamling af data}
Ved at samle data ind vha. 'survey'-modulet findes patientens stemningsleje.
Sideløbende skal de ikke-forstyrrende dataindsamlings-moduler også indsamle data.
Det er så muligt at sammenligne data og sige noget om de ikke-forstyrrende data.

Vi kan ikke på nuværende tidspunkt sige om der er en tydelig sammenhæng mellem de to datakilder. 
Ønsket er at kunne konstruere en model, på baggrund af et eksperiment eller flere eksperimenter.
Modellen skal ud fra ikke-forstyrrende indsamlet data om patientens sociale aktivitet kunne give et bud på patientens stemningsleje.\bruno{Disse tre sætninger måske bare refleksion? Eller måske gøre klar til det her i det nye afsnit 'problemformulering'.}