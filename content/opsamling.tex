\chapter{Opsamling}
Dette projekt har søgt at etablere en platform til udførelse af eksperimenter, der kan være med til at beskrive eventuelle relationer mellem forskellige datakilder indsamlet på en Android smartphone.
Denne er baseret på en platform\citefaelles{} udviklet i samme projektperiode, i samarbejde med en anden projektgruppe.
Fokus har været på at understøtte eksperimenter der undersøger relationen mellem social aktivitet (sms, opkald etc.) og stemningsleje (baseret på patientens besvarelse af spørgsmål).
Hertil er to ikke-forstyrrende\footnote{For definition se \cref{begreber::ikke_forstyrrende}} data moduler (sms/mms og opkald) blevet implementeret.

Da det ikke på forhånd kan vurderes hvilke informationer der relaterer sig til social aktivitet, er en generel løsning blevet implementeret.
Denne giver samtidig anledning til undersøgelse af andre relationer end netop den sociale.
For understøttelse af indsamling af oplysninger fra patienten er et forstyrrende\footnote{For definition se \cref{begreber::forstyrrende}} data modul implementeret.

%I det følgende gives en opsamling på projektets produkt, samt de problemstillinger og mål der blev beskrevet i \cref{problem}.

\section{Relation mellem social aktivitet og stemningsleje}
Dette afsnit beskriver kort de implementerede moduler og deres relation til udførelse af et eksperiment, der skal hjælpe med at afklare relationen mellem social aktivitet og stemningsleje.
Yderligere gives en beskrivelse af implementationen af moduler i et brede perspektiv.
Herved opsummeres også på de første to punkter i målsætningerne for projektet, listet i \cref{problem:maal}.
Dette vil udgøre den anden del af det indsamlede data der skal sammenlignes i eksperimentet.

I \cref{refleksion} diskuteres de aspekter af målsætningerne der ikke blev opnået i projektforløbet.
Her findes også den diskussion (se \cref{refleksion:usikkerhed}) af problemstillinger i forbindelse med udførelse af eksperimenter, som var beskrevet i disse målsætningerne.

\subsection{Ikke-forstyrrende}
En række mulige ikke-forstyrrende (se \cref{ikke_forstyrrende}) datakilder er blevet vurderet for deres mulige relation til social aktivitet.
Af disse er to blevet implementeret som selvstændige moduler (se \cref{implementerede_moduler:opkald} \nameref{implementerede_moduler:opkald} og \cref{implementerede_moduler:smsmms} \nameref{implementerede_moduler:smsmms}).
De to moduler gør det muligt at indsamle ikke-forstyrrende data om patientens sociale aktivitet.
Moduler er implementeret som data moduler og har dermed til formål kun at indsamle informationer.
Det vil sige at de ikke foretager nogen analyse af den indsamlede information.

\subsection{Forstyrrende}
I \cref{maaling_af_stemningsleje} er der fundet frem til tre \textit{spørgeskemaer} til bestemmelse af en patients stemningsleje.
For at kunne sikre at hvert af disse kan understøttes, er de blevet undersøgt med henblik på at klarlægge eventuelle behov.
Disse er beskrevet i \cref{krav}.
Behovene har været grundstenen for det implementerede data modul 'survey', beskrevet i \cref{survey}.

Modulet gør det muligt at foretage forstyrrende dataindsamling fra patienten omkring hans stemningsleje.
Dette kan ske ved hjælp af tre spørgsmåls-typer.
Disse tre typer adskiller sig ved den type af besvarelse de hver især accepterer:
\begin{itemize}
\item \textbf{Multiple choice:} Giver mulighed for at patienten vælger sit/sine svar fra en række prædefinerede muligheder.
\item \textbf{Skala:} Giver mulighed for at vælge en værdi på en prædefineret numerisk skala.
En sådan skala kan præsenteres for brugeren med en tekstuel beskrivelse af hver af skalaens ender.
\item \textbf{Tekst:} Lader patienten angive et svar i \textit{''fritekst''}.
Denne type besvarelse kan eksempelvis anvendes til at føre dagbog.
\end{itemize}
Målsætningen om at lade modulet administrere tidsstyring for spørgsmål er kun delvist implementeret.
I \cref{refleksion:tidspunkt} diskuteres problemstillingerne for den nuværende implementation.
\bruno{Måske lidt for detaljeret}
\subsection{Indsamling af data}
Ved at samle data ind vha. 'survey'-modulet findes patientens stemningsleje.
Sideløbende indsamles også data fra de ikke-forstyrrende dataindsamlings-moduler.
Ud fra den indsamlede information vil det forhåbentligt være muligt at etablere en relation mellem de to informationskilder.
\bruno{Bind sammen med \cref{problem}}
