%Til survey: Tilføj "huskekort" - som får patienten til at vykel en tur, yoga etc. - enten at appen selv finder ud af det eller at det er spykologen der løbende kan lægge dem ind.

\section{Forbedringer til 'Survey'-modulet}
Under udviklingen af det nye modul har projektgruppen observeret en række problemstillinger.
Disse problemstillinger vil blive diskuteret i denne sektion, hvor foreslåede løsninger også vil blive præsenteret.

Disse problemstillinger hindrer ikke brugen af modulet, idet kernefunktionaliteten er blevet implementeret i projektforløbet.
Dog bør disse håndteres inden, eller som en del af, udførelse af eksperimenter med modulet bør foretages.

%Holde styr på om man svarer på spørgsmål
\subsection{Manglende svar}
Det udviklede modul er bygget ud fra idéen om en relation mellem spørgsmål og svar.
Det vil sige, at der for hvert spørgsmål forventes et svar.
Dog er dette ikke nødvendigvis altid tilfældet.
Antageligvis vil der opstå situationer hvor patienten ikke besvarer et spørgsmål, altså som det er nu vil det sige at patienten ikke vælger en svarmulighed eller skriver noget tekst og/eller ikke trykker 'ok'.
En mulig årsag til dette kunne være, at patienten grundet depression ikke kan overskue at tage stilling til spørgsmål.
Men også helt almindelige dagligdags situationer som fx. at have glemt telefonen hjemme, eller at den er løbet tør for strøm, kan resultere i at patienten ikke besvarer et spørgsmål.

Det implementerede modul understøtter ikke indsamlingen af information om hvorvidt et spørgsmål er blevet besvaret eller ej.
Det er en mulighed at besvarelser kan være en indikator på en ændring i adfærden, idet patienten pga. manglende overskud kan vælge ikke at besvare.
Det vil naturligvis være nødvendigt at foretage et eksperiment der kan afklare om dette er tilfældet.
For at udføre dette eksperiment vil det være nødvendigt at lade systemet registrere hvilke spørgsmål der er besvaret, samt hvilke der ikke er det.

\paragraph{Aktiv udsættelse af besvarelse}
Som et led i understøttelsen af ovenstående vil det være fordelagtigt, at give patienten selv muligheden for at afvise at besvare et spørgsmål.
Således kan det undersøges om patienten tager et aktivt fravalg af besvarelse, og yderlige kunne angives hvad grunden er for dette (ikke tid, ikke lyst, ved ikke).
Det vil naturligvis ikke være endegyldigt, da et aktivt fravalg vil kunne opfattes passivt af mobiltelefonen.
Patienten kan eksempelvis vælge slet ikke at reagere på en notifikation.
\mikael{Forstår ikke de sidste 2 sætninger her}

%Tilføj "Postpone"-knap så det er muligt at svare senere på spørgsmål
Udover muligheden for at afvise spørgsmål, vil patienten i nogle situationer muligvis blot have brug for en udsættelse af spørgsmålet.
Med denne mulighed kan der \textit{muligvis} indsamles mere information end ved en model der kun har mulighed for komplet afvisning.
Hvorvidt dette faktisk er tilfældet kan undersøges ved at stille funktionaliteten til rådighed for patienten.

%Mere dynamisk styring af hvad tidspunkt spoergsmaal bliver vist paa
\subsection{Tidspunkt for spørgsmål}
\mikkelin{Vi mangler noget tekst til ''Mere dynamisk styring af hvad tidspunkt spoergsmaal bliver vist paa''.}

%GUI skal forbedres meget - fordi det ikke har været i fokus
\mikkelin{Jeg ved ikke hvad jeg skal skrive om ''GUI skal forbedres meget''. Det virker nemt som en standard ting at sige.}

%Anvendelse (model/eksperimentering)
\section{Usikkerheder og eksperimenter}
Forståelsen af relationen mellem patientens adfærd, stemningsleje og interaktion med mobiltelefon er stadig uklar.
Dette projekt lægger op til en række eksperimenter der kan være med til at afklare nogle af disse relationer.
Eksempelvis betydningen af patientens digitale sociale interaktion i forhold til stemningsleje.
Nogle af disse eksperimenter vil kunne udføres simultant, i form af dataindsamlingen.
Efter en periode med indsamling af data, vil det være muligt at begynde at lede efter sammenhænge i det indsamlede data.
Fx relationen mellem stemningsleje og størrelsen af en patients kontaktflade før, under og efter en depression eller manisk periode.

\paragraph{Ikke-eksperimentel anvendelse}
Udover muligheden for at vurdere sammenhænge mellem det indsamlede data kan informationerne også anvendes direkte som en datakilde, på lige fod med telefonens sensorer.
På denne måde kan patienten skabe sig et overblik over sin egen sindstilstand (ex stemningsleje og dagbog) for på den måde objektivt at kunne følge egne subjektive vurderinger over tid.
Altså er det på den måde muligt at eliminere dårlig hukommelse som fejlkilde, og derved skabe et mere realistisk verdensbillede for patienten.
\bruno{\textbf{IVAN:} Afsnit 7.3 kan udvides enormt, hvis I har tid.
Man kan diskutere eksperimenter sygdom for sygdom.
Man kan eksperimentere med forskellige datakilder.
Man kan diskutere validitet af data.
Man kan diskutere tilrettelæggelse af eksperimenter og eksperimentforløb.
Dette afsnit og kapitel 4-5 kan løfte rapporten væsentligt}

\section{title}
\bruno{\textbf{IVAN:} Kunne man forestille
sig at Scheduler
prøver at finde stille
stunder, hvor det er
sandsynligt at 
personen er med på
at svare på spørgsmål?
}
\mikael{Passer det her ikke fint ind under 7.1.2 (Tidspunkt for spørgsmål)?}