%Til survey: Tilføj "huskekort" - som får patienten til at vykel en tur, yoga etc. - enten at appen selv finder ud af det eller at det er spykologen der løbende kan lægge dem ind.
\section{Notifikationer}\bruno{Skal flyttes til fælles, fordi det handler mere om platform end om vores modul.}
En naturlig udvikling af det udviklede survey modul ville være, i stedet for at indhente oplysninger fra brugeren, at præsentere dem.
Altså at kunne notificere brugeren med information.
Dette kunne enten ske på faste tidspunkter eller ved at et analyse modul registrerer en ændring i et af de monitorerede adfærdsmønstre.
Ligeledes kan patienten modtage en status-notifikation, der fx kan beskrive en opsummering over hans sociale aktivitet målt over den seneste uge.

\paragraph{Det udviklede modul}
Survey modulet tillader at indsamle information fra brugeren via en række forskellige typer spørgsmål (se \cref{survey:spg}).
Modulet fungerer herved som et sensor modul der anvender brugerens respons som dets input.
Modulet adskiller sig dog fra andre sensor moduler idet det anvender en grafisk brugergrænseflade til at indsamle information.
Givet relationen mellem de forskellige moduler (sensor-, analyse- og visningsmoduler) introducerer ovenstående en ny problemstilling.

Da information kun kan overføres fra sensor til analyse og videre til visning, og ikke i den modsatte retning, vil notifikationer skulle implementeres som et visningsmodul.
Kun på denne måde kan en notifikation præsentere information fra underliggende lag.
Det betyder at notifikationer og spørgsmål vil skulle udvikles som to separate moduler.
Modulerne kan defineres som en del af samme applikation for derved at kunne anvende samme kode-base, men vil i sidste ende skulle fungere som to moduler.

Skulle man yderligere ønske at præsentere information i et spørgsmål ville det kræve en sammensmeltning af de to typer moduler.
En sådan kombination af moduler er problematisk under den nuværende arkitektur, da den vil kræve både input og output i samme modul.

\section{Forbedringer til 'Proof of Concept' - modulet}
Under udviklingen af det nye modul har projektgruppen observeret en række problemstillinger.
Disse hindrer ikke brugen af modulet, idet kernefunktionaliteten er blevet implementeret i projektforløbet.
Dog bør disse håndteres inden, eller som en del af, udførelse af eksperimenter med modulet.

Dette afsnit diskuterer nogle af udfordringerne forbundet med videreudviklingen af det udviklede survey modul.

%Holde styr på om man svarer på spørgsmål
\subsection{Manglende svar}
Det udviklede modul er bygget op om ideen om relationen mellem spørgsmål og svar.
Det vil sige, at der for hvert spørgsmål forventes et svar.
Dog er dette ikke nødvendigvis altid tilfældet.
Man må antage at der opstår situationer hvor patienten ikke besvarer et spørgsmål.
En mulig årsag til dette kunne være, at patienten grundet depression ikke kan overskue at tage stilling til spørgsmål.
Men også helt almindelige dagligdags situationer som fx at have glemt telefonen hjemme eller løbe tør for strøm, kan resultere i at patienten ikke besvarer et spørgsmål.

Det implementerede modul understøtter ikke indsamlingen af information om hvorvidt et spørgsmål er blevet besvaret eller ej.
Da manglende besvarelse kan være en mulig indikation for en kommende depression, idet patienten pga manglende overskud kan finde på ikke at besvare spørgsmål.
Det vil naturligvis være nødvendigt at foretage et eksperiment der kan afklare om en sådan relation eksisterer.
Men for at udføre dette eksperiment vil det være nødvendigt at lade systemet registrere hvilke spørgsmål der er besvaret, samt hvilke der ikke er det.

\paragraph{Aktiv udsættelse af besvarelse}
Som et led i understøttelsen af ovenstående vil det være fordelagtigt, at give patienten selv muligheden for at afvise at besvare et spørgsmål.
Således kan det undersøges om patienten tager et aktivt fravalg af besvarelse.
Det vil naturligvis ikke være endegyldigt, da et aktivt fravalg vil kunne opfattes passivt af mobiltelefonen.
Patienten kan eksempelvis vælge slet ikke at reagere på en notifikation.

%Tilføj "Postpone"-knap så det er muligt at svare senere på spørgsmål
Udover muligheden for at afvise besvarelse, vil patienten i nogle situationer muligvis blot have brug for en udsættelse af besvarelsen.
Med denne mulighed kan der \textit{muligvis} indsamles mere information end ved en model der kun har mulighed for komplet afvisning.
Hvorvidt dette faktisk er tilfældet, kan på undersøges ved at stille funktionaliteten til rådighed for patienten.

%Mere dynamisk styring af hvad tidspunkt spoergsmaal bliver vist paa
\subsection{Tidspunkt for spørgsmål}
\mikkelin{Vi mangler noget tekst til ''Mere dynamisk styring af hvad tidspunkt spoergsmaal bliver vist paa''.}

%GUI skal forbedres meget - fordi det ikke har været i fokus
\mikkelin{Jeg ved ikke hvad jeg skal skrive om ''GUI skal forbedres meget''. Det virker nemt som en standard ting at sige.}

%Anvendelse (model/eksperimentering)
\section{Usikkerheder og eksperimenter}
Forståelsen af relationen mellem patientens adfærd, stemningsleje og interaktion med mobiltelefon er stadig uklar.
Dette projekt lægger op til en række eksperimenter der kan være med til at afklare nogle af disse relationer.
Eksempelvis betydningen af patientens digitale sociale interaktion i forhold til stemningsleje.
Nogle af disse eksperimenter vil kunne udføres simultant, i form af dataindsamlingen.
Efter en periode med indsamling af data, vil det være muligt at begynde at lede efter sammenhænge i det indsamlede data.
Fx relationen mellem stemningsleje og størrelsen af en patients kontaktflade før, under og efter en depression eller manisk periode.

\paragraph{Ikke-eksperimentel anvendelse}
Udover muligheden for at vurdere sammenhænge mellem det indsamlede data kan informationerne også anvendes direkte som en datakilde, på lige fod med telefonens sensorer.
På denne måde kan patienten skabe sig et overblik over sin egen sindstilstand (ex stemningsleje og dagbog) for på den måde objektivt at kunne følge egne subjektive vurderinger over tid.
Altså er det på den måde muligt at eliminere dårlig hukommelse som fejlkilde, og derved skabe et mere realistisk verdensbillede for patienten.
\bruno{\textbf{IVAN:} Afsnit 7.3 kan udvides enormt, hvis I har tid.
Man kan diskutere eksperimenter sygdom for sygdom.
Man kan eksperimentere med forskellige datakilder.
Man kan diskutere validitet af data.
Man kan diskutere tilrettelæggelse af eksperimenter og eksperimentforløb.
Dette afsnit og kapitel 4-5 kan løfte rapporten væsentligt}

\section{title}
\bruno{\textbf{IVAN:} Kunne man forestille
sig at Scheduler
prøver at finde stille
stunder, hvor det er
sandsynligt at 
personen er med på
at svare på spørgsmål?
}