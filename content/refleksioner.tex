%Til survey: Tilføj "huskekort" - som får patienten til at vykel en tur, yoga etc. - enten at appen selv finder ud af det eller at det er spykologen der løbende kan lægge dem ind.
\label{refleksion}
\mikkel{Skriv at det hele er lidt (\textbf{MEGET}) løst}
Dette kapitel diskuterer projektgruppens løsning ved at kigge på forbedringsforslag og beskrive hvilke overvejelser det har ført til.
Først kigges der på forbedringer til 'survey'-modulet, som især belyser de overvejelser der er gjort ifht. hvad tidspunkt spørgsmål skal stilles.
Derefter belyses og diskuteres de overvejelser der er gjort ifht. de usikkerheder der er ved dataindsamling, samt validering af den indsamlede data.

\section{Forbedringer til 'survey'-modulet}
Under udviklingen af survey modulet, beskrevet i \cref{survey}, har projektgruppen observeret en række problemstillinger.
Disse vil blive diskuteret i denne sektion, hvor foreslåede løsninger også vil blive præsenteret.

Problemstillingerne hindrer ikke brugen af modulet, idet kernefunktionaliteten er blevet implementeret i projektforløbet.
Dog bør de håndteres inden udførelse af eksperimenter med modulet.

%Holde styr på om man svarer på spørgsmål
\subsection{Manglende besvarelser}\label{reflection:manglende}
Det udviklede modul er bygget ud fra idéen om en relation mellem spørgsmål og svar.
Det vil sige, at der for hvert spørgsmål forventes et svar.
Dog er dette ikke nødvendigvis altid tilfældet.
Antageligvis vil der opstå situationer hvor patienten ikke besvarer et spørgsmål.
Som systemet er nu vil det sige at patienten ikke indtaster et svar og trykker 'ok', eller slet ikke trykker 'ok'.
En mulig årsag til dette kunne være at patienten grundet lavt stemningsleje ikke kan overskue at tage stilling til spørgsmål.
Ligeledes kan mere almindelige situationer resultere i at patienten ikke besvarer et spørgsmål.
Eksempelvis kan han have efterladt telefonen hjemme eller den kan være løbet tør for strøm.

Det implementerede modul understøtter ikke indsamlingen af meta-information om besvarelsen af spørgsmål, herunder om hvorvidt spørgsmålet overhovedet er blevet besvaret.
Det er en mulighed at besvarelser, eller mangel på samme, kan være en indikator på en ændring i adfærden, idet patienten pga. manglende overskud kan vælge ikke at besvare.
Det vil naturligvis være nødvendigt at foretage et eksperiment der kan afklare om dette er tilfældet.
For at udføre dette eksperiment vil det være nødvendigt at lade systemet registrere hvilke spørgsmål der er besvaret, samt hvilke der ikke er det.

\paragraph{Aktiv udsættelse af besvarelse}
Som et led i understøttelsen af ovenstående vil det være fordelagtigt, at give patienten selv muligheden for at afvise at besvare et spørgsmål.
Således kan det noteres om patienten tager et aktivt fravalg af besvarelse, og yderlige kunne angives hvad grunden er for dette (ikke tid, ikke lyst, ved ikke).
Det vil dog ikke dække alle situationer, da det også kunne forestilles at patienten vælger slet ikke at reagere på en notifikation.
Her kunne det så vælges at automatisk fjerne spørgsmålet efter noget tid og notere at dette skete, dog kræver dette flere overvejelser, da patienten blot kunne være væk fra telefonen eller tager særligt lang tid om at svare.

%Tilføj "Postpone"-knap så det er muligt at svare senere på spørgsmål
Udover muligheden for at afvise spørgsmål, vil patienten i nogle situationer muligvis blot have brug for en udsættelse af spørgsmålet.
Med denne mulighed kan der \textit{muligvis} indsamles mere information end ved en model der kun har mulighed for komplet afvisning.
Hvorvidt dette faktisk er tilfældet kan undersøges ved at stille funktionaliteten til rådighed for patienten.

%Mere dynamisk styring af hvad tidspunkt spoergsmaal bliver vist paa
\subsection{Tidspunkt for spørgsmål}\label{refleksion:tidspunkt}
I \cref{tidspnkforspg} blev der præsenteret nogle overvejelser der er gjort i forhold til hvornår spørgsmål bliver stillet.
Da der er mange spørgsmål at stille fra de omtalte metoder (se \cref{datasamling}) er det vigtigt at overveje hvordan man undgår at overdænge patienten med dem alle på én gang.
Det implementerede system tillader kun begrænset tilpasning af hvornår spørgsmål bliver stillet - tidspunkt på dagen og størrelse på interval.
En mere dynamisk løsning vil være at gruppere spørgsmål og gøre det muligt at indstille hvordan og hvornår denne gruppe spørgsmål skal stilles til patienten.

Indstillingsmuligheder for en gruppe af spørgsmål kunne være foretage en \emph{reduktion} i antallet af samtidige spørgsmål.
Dette vil sige at der er et maksimalt antal af spørgsmål der skal blive stillet over en bestemt tidsperiode, eller på samme tidspunkt.
Grunden til at foretage denne reduktion i spørgsmål er at patienten kunne lave forhastede besvarelser, hvis der stilles for mange spørgsmål og for ofte.
Eksempelvis kunne man for PANAS vælge maksimalt at stille 4 spørgsmål på én gang, samt at hvert spørgsmål kun stilles én gang pr. dag.

Da det stadig ønskes at patienten besvarer samtlige spørgsmål, kan der derudover foretages en \emph{variation} over de benyttede spørgsmål.
Hvis patienten dagligt får de samme spørgsmål, i samme rækkefølge, kunne dette lede til rutineprægede besvarelser.
Der skal altså være enten en tilfældig eller justerbar variation i de spørgsmål der stilles.
For PANAS kunne dette være at, hvis der stilles spørgsmål i grupper af 4, at disse grupper er tilfældigt dannet, eller at de 4 spørgsmål bliver stillet i tilfældig rækkefølge.

%GUI skal forbedres meget - fordi det ikke har været i fokus
\subsection{Brugergrænseflade}
Som nævnt i \cref{survey} bruger surveymodulet standard komponenter fra Android-bibilioteket til brugergrænsefladen.
Dette betyder at brugergrænsefladen i flere tilfælde er meget simpel og ikke særlig hensigtsmæssig for brugere.
Et eksempel på dette kan ses på \cref{stemreg::notater_vaegt} hvor tekstfeltet er kun er en enkelt linje.
Dette vil ikke være en god visning for indtastning af dagbog.
Inden surveymodulet skal bruges i et eksperiment bør man derfor forbedre grænsefladen ved at lave tilpassede layouts til hvert af spørgsmålene.


%Anvendelse (model/eksperimentering)
\section{Usikkerheder og eksperimenter}\label{refleksion:usikkerhed}
Forståelsen af relationen mellem patientens adfærd, stemningsleje og interaktion med mobiltelefon er stadig uklar.
Dette projekt lægger op til en række eksperimenter der kan være med til at afklare nogle af disse relationer.
Eksempelvis betydningen af patientens digitale sociale interaktion i forhold til stemningsleje.

Dataindsamlingen til nogle af disse eksperimenter vil kunne udføres simultant.
Enten fordi samme data anvendes til flere eksperimenter eller fordi to kilder er hinanden uafhængige.
Eksempelvis er indsamlingen af oplysninger om sms uafhængig af oplysninger om opkald.
Bemærk at brugerens brug af kilder (eksempelvis sms og opkald) ikke nødvendigvis deler samme uafhængighed.
Altså kunne flere afsendte sms beskeder muligvis resultere i et fald i antallet af opkald.

Efter en periode med indsamling af data, vil det være muligt at begynde at lede efter sammenhænge i den mængde data der er blevet indsamlet.
Fx. relationen mellem stemningsleje og størrelsen af en patients sociale netværk før, under og efter en depression eller manisk periode.

I de følgende afsnit diskuteres overvejelser der bør inddrages i forbindelse med udførelse af eksperimenter samt forslag til brug udenfor eksperimentel kontekst.

\subsection{Sammenligning på tværs af sygdomme}\label{refleksion:tvaers_af_sygdomme}
Ud over en generel forbindelse mellem stemningsleje og social aktivitet, kunne der også lægges vægt på bestemte sygdomme.
Som udgangspunkt er målgruppen for dette projekt patienter diagnosticeret med en unipolar eller bipolar affektiv lidelse.
Dog kunne patienter, diagnosticeret med stress og lettere depressioner, ligeledes være interessante for projektet.

Ved at tage en større gruppe af patienter med samme sygdom, kunne indsamlet data sammenlignes på tværs af disse.
Hvis dette samtidig gøres for flere forskellige sygdomme, kan disse sammenlignes på tværs.
På denne måde vil det være muligt at opbygge en model for hver sygdom, eller muligvis blot konstatere at en model ikke kan konstrueres ud fra den mængde data der er blevet indsamlet.

En model der på denne måde beskriver en generalisering af patienter der alle lider af en bestemt sygdom vil kunne anvendes til analyse af patienter for hvilke en norm endnu ikke er etableret.
Således vil det være muligt at genanvende informationer fra tidligere/eksisterende patienter til nye brugere af systemet.

\subsection{Sammenligning på tværs af sensorer/analyser}
Ligesom sammenligning på tværs af sygdomme, kunne det være interessant at eksperimentere med brug af flere forskellige sensorer og/eller analyser.
På denne måde kan den enkelte sensor evalueres i relation til andre lignende sensorer.
Altså kunne der eksempelvis laves et modul til bestemmelse af lokation baseret på GPS og et tilsvarende baseret på WiFi.
For disse kan det eksempelvis undersøges hvilken af de to der angiver den mest præcise lokation.
Ligeledes bør det undersøges hvilken data-kilde der mest pålideligt udtaler sig om en patients sociale aktivitet.
Generelt er det interessant at vurdere pålideligheden af informationen fra moduler, præcisionen samt frekvensen af data.

Ved sammenligning af to moduler vil den manipulation af data der foretages af modulet påvirke resultatet af en sammenligning af to moduler.
Det antages at et analyse modul typisk vil foretage mere manipulation af data end et data modul.
Eksempelvis kan et analyse modul være et der samler informationer fra flere kilder og giver et samlet resultat.
Et eksempel på et sådan modul kunne være et der aflæser informationer fra de to implementerede ikke-forstyrrende moduler (se \cref{implementerede_moduler}) og kombinerer dem til en samlet vurdering af social aktivitet.

Det største problem med denne form for sammenligning er at det indsamlede sensor/analyse data skal evalueres i forhold til de tre kriterier; pålidelighed, præcision og frekvens.
Altså sensorens/analysens evne til at \textit{afspejle virkeligheden}.
Denne vurdering må foretages på basis af det enkelte modul, da typen af data varierer meget fra modul til modul.

\subsection{Validitet af data}
Som nævnt er det relevant at vurdere de forskellige datakilder, før de kan anvendes i sammenligning med andre datakilder.
Pålideligheden af en kilde kan vurderes ved at undersøge en række egenskaber ved kilden.
Nogle kilder lever kun delvis data mens andre er afhængige af et ekstern stykke hardware.
Eksempelvis har GPS ingen mangelfuld information, da systemet kan anvendes overalt.
Det har derimod en afhængighed af ekstern hardware, i form a satellitter.
I modsætning er bruger input ikke afhængig af anden hardware (udover patienten selv), men her kan informationen være mangelfuld (se reflection:manglende).

I nogle tilfælde vil det være muligt at undgå disse hindringer ved at supplere med data fra anden kilde.

For mange datakilder vil både præcision og frekvens være oplyste fra producenten.
Alternativt kan der nemt opstilles eksperimenter der vurderer disse, givet at der findes en måde at måle præcisionen af data fra en kilde.
Eksempelvis kan præcisionen af GPS måles ved afstanden mellem to returnerede punkter og den faktiske afstand mellem dem.
Nogle kilder kan ikke vurderes på denne måde, som beskrevet i \cref{reflection:valid_resultater}.

\paragraph{Manglende data}
Der er også sværere overvejelser, som hvordan man kan konkludere på det indsamlede data.
Ved store perioder uden ny data, for eksempel ingen nye opkald eller SMSer, vil dette ikke nødvendigvis betyde at brugeren har isoleret sig, men blot at mobiltelefonen er glemt hjemme eller løbet tør for strøm.
For at øge præcisionen af brugen af denne data, kunne det være nødvendigt at supplere med andre ikke-relaterede datakilder (fx. lokation og accelerometer til at sige noget om sandsynligheden for at mobilen er på personen).

\subsection{Validering af resultater}\label{reflection:valid_resultater}
%ref til gruppeinterview ifm. interesse i egen forbedring
Dette afsnit beskæftiger sig med overvejelser vedr. validering af data fra udførte eksperimenter.
Her tages der udgangspunkt i et tænkt eksempel på et eksperiment, hvor der løbende indsamles information om SMS (se \cref{implementerede_moduler:smsmms}) og flere gange dagligt laves forespørgsler baseret på PANAS (se \cref{PANAS}).

Baseret på informationerne fra eksperimentet vurderes det om de to kilder indikerer den samme information.
Eksempelvis hvorvidt de indikerer den samme relation mellem social aktivitet og stemningsleje.
Her er det væsentligt at bemærke, at dette ikke kan fastslå hvor sandfærdige kilderne er.
Altså vil man ikke kunne sige noget om relationen mellem social aktivitet og stemningsleje, men kun hvorvidt de to datakilder giver den samme vurdering eller ej.

Problematikken opstår da ingen kendte metoder findes til objektivt at fastslå stemningsleje eller social aktivitet, se \cref{begreber::forstyrrende}.
Altså kendes den endegyldige sandhed\footnote{Engelsk: ground truth.} ikke og resultaterne kan derved ikke objektivt valideres for deres sandhed.
De kan dog alternativt sammenholdes med en psykologs/psykiaters vurdering.
\Citet{goldberg} beskriver et eksperiment, der har anvendt netop denne metode.
I eksperimentet blev et spørgeskema, bestående af 60 spørgsmål ang. generelt psykisk helbred, sammenlignet med en psykiatrisk vurdering.
Dette blev gjort med en dobbelt-blind test, hvor 200 patienter ved to lægekonsultationer først udfyldte spørgeskemaet, hvorefter patienten fik foretaget en psykiatrisk vurdering af lægen, hvor lægen ikke så på resultatet af spørgeskemaet før hans egen vurdering var givet.
På denne måde blev patienter vurderet ud fra spørgeskemaet og lægens vurdering hver især, og disse resultater blev holdt op imod hinanden.
Eksperimentet viste at over 90\% af patienterne blev klassificeret korrekt af spørgeskemaet.

Et senere eksperiment af ovennævnte type kunne udføres, til at validere konklusion baseret på tidligere sensor/analyse og survey dataindsamling.
Dette vil kræve en større mængde patienter, samt indgående kendskab til tidligere diagnoser.
Herved øges sandsynligheden for at en patient vil blive ramt af en periode (da der kan gå flere år mellem perioder), og resultatet af det indsamlede data vil kunne holdes op imod en psykiatrisk vurdering.

\subsection{Ikke-eksperimentel anvendelse}
Udover muligheden for at vurdere sammenhænge mellem det indsamlede data kan informationerne også anvendes direkte som en datakilde, på lige fod med telefonens sensorer.
På denne måde kan patienten skabe sig et overblik over sin egen stemningsleje (fx. stemningsregistrering, \cref{stemningsleje::stemningsregistrering}) for på den måde objektivt at kunne følge egne subjektive vurderinger over tid.
Altså er det på den måde muligt at eliminere dårlig hukommelse som fejlkilde, og derved skabe et mere realistisk verdensbillede for patienten.
