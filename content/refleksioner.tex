%Til survey: Tilføj "huskekort" - som får patienten til at vykel en tur, yoga etc. - enten at appen selv finder ud af det eller at det er spykologen der løbende kan lægge dem ind.

\bruno{Split måske i diskussio  og refleksion}

\section{Forbedringer til 'survey'-modulet}
Under udviklingen af survey modulet, beskrevet i \cref{survey}, har projektgruppen observeret en række problemstillinger.
Disse vil blive diskuteret i denne sektion, hvor foreslåede løsninger også vil blive præsenteret.

Problemstillingerne hindrer ikke brugen af modulet, idet kernefunktionaliteten er blevet implementeret i projektforløbet.
Dog bør de håndteres inden udførelse af eksperimenter med modulet.

%Holde styr på om man svarer på spørgsmål
\subsection{Manglende besvarelser}
Det udviklede modul er bygget ud fra idéen om en relation mellem spørgsmål og svar.
Det vil sige, at der for hvert spørgsmål forventes et svar.
Dog er dette ikke nødvendigvis altid tilfældet.
Antageligvis vil der opstå situationer hvor patienten ikke besvarer et spørgsmål.
Altså som systemet er nu vil det sige at patienten ikke vælger en svarmulighed eller skriver noget tekst, og/eller ikke trykker 'ok'.
En mulig årsag til dette kunne være at patienten grundet lavt stemningsleje ikke kan overskue at tage stilling til spørgsmål.
Dog også mere almindelige situationer, som fx. at have glemt telefonen hjemme eller at den er løbet tør for strøm, kan resultere i at patienten ikke besvarer et spørgsmål.

Det implementerede modul understøtter ikke indsamlingen af meta-information om besvarelsen af spørgsmål, herunder om hvorvidt spørgsmålet overhovedet er blevet besvaret.
Det er en mulighed at besvarelser, eller mangel på samme, kan være en indikator på en ændring i adfærden, idet patienten pga. manglende overskud kan vælge ikke at besvare.
Det vil naturligvis være nødvendigt at foretage et eksperiment der kan afklare om dette er tilfældet.
For at udføre dette eksperiment vil det være nødvendigt at lade systemet registrere hvilke spørgsmål der er besvaret, samt hvilke der ikke er det.

\paragraph{Aktiv udsættelse af besvarelse}
Som et led i understøttelsen af ovenstående vil det være fordelagtigt, at give patienten selv muligheden for at afvise at besvare et spørgsmål.
Således kan det noteres om patienten tager et aktivt fravalg af besvarelse, og yderlige kunne angives hvad grunden er for dette (ikke tid, ikke lyst, ved ikke).
Det vil dog ikke dække alle situationer, da det også kunne forestilles at patienten vælger slet ikke at reagere på en notifikation.
Her kunne det så vælges at automatisk fjerne spørgsmålet efter noget tid og notere at dette skete, dog kræver dette flere overvejelser, da patienten blot kunne være væk fra telefonen eller tager særligt lang tid om at svare.

%Tilføj "Postpone"-knap så det er muligt at svare senere på spørgsmål
Udover muligheden for at afvise spørgsmål, vil patienten i nogle situationer muligvis blot have brug for en udsættelse af spørgsmålet.
Med denne mulighed kan der \textit{muligvis} indsamles mere information end ved en model der kun har mulighed for komplet afvisning.
Hvorvidt dette faktisk er tilfældet kan undersøges ved at stille funktionaliteten til rådighed for patienten.

%Mere dynamisk styring af hvad tidspunkt spoergsmaal bliver vist paa
\subsection{Tidspunkt for spørgsmål}
\mikkelin{Vi mangler noget tekst til ''Mere dynamisk styring af hvad tidspunkt spoergsmaal bliver vist paa''.}

%GUI skal forbedres meget - fordi det ikke har været i fokus
\subsection{Brugergrænseflade}
Som nævnt i \cref{survey} bruger survey standard komponenter fra Android-bibilioteket til brugergrænsefladen.
Brugergrænsefladen er ikke blevet testet fordi den ikke har været i fokus.
Den skal testes for brugervenlighed, da dette ikke er blevet gjort fx vha. en test i usability-lab.\bruno{god usability-lab kilde!?!?!?!?!?! anyone?}
\mikkelin{Jeg ved ikke hvad jeg skal skrive om ''GUI skal forbedres meget''. Det virker nemt som en standard ting at sige.}

%Anvendelse (model/eksperimentering)
\section{Usikkerheder og eksperimenter}
Forståelsen af relationen mellem patientens adfærd, stemningsleje og interaktion med mobiltelefon er stadig uklar.
Dette projekt lægger op til en række eksperimenter der kan være med til at afklare nogle af disse relationer.
Eksempelvis betydningen af patientens digitale sociale interaktion i forhold til stemningsleje.
Nogle af disse eksperimenter vil kunne udføres simultant, i form af dataindsamlingen.
Efter en periode med indsamling af data, vil det være muligt at begynde at lede efter sammenhænge i det indsamlede data.
Fx. relationen mellem stemningsleje og størrelsen af en patients sociale netværk før, under og efter en depression eller manisk periode.

Her vil blive diskuteret overvejelser der kan gøres omkring eksperimentering, og til sidst forslag til brug udenfor eksperimentel kontekst.

\subsection{Sammenligning på tværs af sygdomme}
Ud over en generel forbindelse mellem stemningsleje og social aktivitet, kunne der også lægges vægt på bestemte sygdomme.

Som udgangspunkt er målgruppen for dette projekt patienter diagnosticeret med unipolar eller bipolar affektiv lidelse.
Dog kunne andre interessante sygdomme være stress og lettere depressioner.
Det kunne også være interessant at nærmere undersøge de forskellige typer af affektive lidelser på tværs, for at finde en generel forbindelse, der kunne bruges til lettere at danne model for den enkelte patient.
Dette vil kræve eksperimenter med patienter diagnosticeret med en af de førnævnte lidelser, samt viden om hvorvidt patienten er før/under/efter en periode.

Ved at tage en større gruppe af patienter med samme sygdom, kunne den indsamlede data sammenlignes på tværs af disse.
Hvis dette samtidig gøres for flere forskellige sygdomme, kan disse sammenlignes på tværs.
Når man så opnår et resultat, kan det vurderes om en diagnose er en brugbar faktor når der skal dannes model for den enkelte patient.

\subsection{Sammenligning på tværs af sensorer/analyser}
Ligesom sammenligning på tværs af sygdomme, kunne det være interessant at eksperimentere med brug af flere forskellige sensorer og/eller analyser.

På denne måde kan den enkelte sensor evalueres individuelt.
For analyser, som nok vil være en form for fusion af sensorer, samt manipulation af disses input, vil dette være en smule mere avanceret.
Dog vil man på samme måde kunne evaluere den individuelle analyse.

Det største problem med denne form sammenligning er at det indsamlet sensor/analyse data skal kunne holdes op mod noget der afspejler virkeligheden, for at kunne sige i hvor høj grad dataet afspejler virkeligheden.
Samtidig, med antagelsen fra før om at der kan være sammenhæng mellem patienter, med samme sygdom, og deres relation mellem stemningsleje og social aktivitet, bliver man nødt til at eliminere denne faktor.
Dette fører til næste punkt.

\subsection{Validering af resultater}
%ref til gruppeinterview ifm. interesse i egen forbedring
Efter et eksperiment er udført og der er kommet frem til nogle resultater, er der stadig nogle overvejelser ift. metode, navnligt validering af resultater.

Eksempelvis, hvis vi antager at der er udført et eksperiment hvor der indsamles løbende SMS og opkald meta-data, samt flere daglige forespørgsler baseret på PANAS.
Hvis det vurderes at der er en god sammenhæng mellem de to dataindsamlinger over tid, skal det vurderes hvorvidt dette resultat er sandfærdigt.
Hvis formålet er at vurdere om hvorvidt SMS og opkald er gode kilder til at sige noget om social aktivitet, skal de holdes op imod noget der sandfærdigt siger noget om patientens aktuelle sociale aktivitet (ground truth).
Dette er dog ikke helt lige til, da det tidligere er vurderet at det ikke er muligt at få helt objektive besvarelser, se \cref{begreber::forstyrrende}.

Umiddelbart er der ingen anden måde hvorpå denne ground truth kan indsamles, det nærmeste er måske en psykologs/psykiaters vurdering.
Et eksperiment der har brugt netop denne metode er \citet{goldberg}.
I dette eksperiment blev et spørgeskema, bestående af 60 spørgsmål ang. generelt psykisk helbred, sammenlignet med en psykiatrisk vurdering.
Dette blev gjort med en dobbelt-blind test, hvor 3.000 patienter ved to lægekonsultationer først udfyldte spørgeskemaet, hvorefter patienten fik foretaget en psykiatrisk vurdering af lægen, hvor lægen ikke så på resultatet af spørgeskemaet før hans egen vurdering var givet.
På denne måde blev patienter vurderet ud fra spørgeskemaet og lægens vurdering hver især, og disse resultater blev holdt op imod hinanden.

Et senere eksperiment af ovennævnte type kunne udføres, til at validere konklusion baseret på tidligere sensor/analyse og survey dataindsamling.
Dette vil kræve en større mængde patienter, samt indgående kendskab til tidligere diagnoser.
På denne måde øges chancen for at ramme patienter der vil blive ramt af en periode (da der kan gå flere år mellem perioder), og resultatet af det indsamlede data vil kunne holdes op imod en psykiatrisk vurdering.

\subsection{Validitet af data}
Når der er indsamlet noget data, er der en række ting der kan overvejes omkring validiteten af denne.

Den overordnede brugbarhed kan overvejes, her tænkes om dataet er dækkende, altså om hvorvidt der er huller, som fx. ved lokation data hvor der kan opstå huller ved dårlig eller ingen GPS/mobil dækning.
Her kan det så overvejes om disse huller kan lappes ved at supplere med data fra anden kilde.

De individuelle data kilder kan også holdes op imod hinanden, for at afgøre om de er enige eller uenige.

Der er også sværere overvejelser, som hvordan man kan konkludere på det indsamlede data.
Ved store perioder uden ny data, for eksempel ingen nye opkald eller SMSer, vil dette ikke nødvendigvis betyde at brugeren har isoleret sig, men blot at mobiltelefonen er glemt hjemme eller løbet tør for strøm.
For at øge præcisionen af brugen af denne data, kunne det være nødvendigt at supplere med andre ikke-relaterede datakilder (fx. lokation og accelerometer til at sige noget om sandsynligheden for at mobilen er på personen).

\subsection{Ikke-eksperimentel anvendelse}
Udover muligheden for at vurdere sammenhænge mellem det indsamlede data kan informationerne også anvendes direkte som en datakilde, på lige fod med telefonens sensorer.
På denne måde kan patienten skabe sig et overblik over sin egen sindstilstand (fx. stemningsregistrering, \cref{stemningsleje::stemningsregistrering}) for på den måde objektivt at kunne følge egne subjektive vurderinger over tid.
Altså er det på den måde muligt at eliminere dårlig hukommelse som fejlkilde, og derved skabe et mere realistisk verdensbillede for patienten.

\section{title}

\stefan{ Vi har skrevet følgende i \cref{4_magiske_ord}:
	Hvis disse spørgsmål kom alle 20 ad gangen, og i samme rækkefølge, ville det være muligt at huske (måske endda ubevist) en rutine i udfyldningen af svarene. - Vi har ikke kilde på det, skriv refleksioner
	}