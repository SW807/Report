\chapter{Begreber}

\section{Affektive Fænomener}\mikael{Overskriften er direkte oversat fra fællesbetegnelsen brugt i kilden, ved ikke om den er god nok.}
Dette afsnit introducerer en række begreber indenfor psykologien.
Definitionen af begreberne er baseret på oversættelser af \citet{ekkekakis}.
\mikkel{Er begreberne bare oversættelser af kilden? For så mener jeg at vi bør vise dem som citater.}

\subsection[Affekt]{Affekt\footnote{Engelsk: affect}}
Affekt beskriver en neurologisk tilstand, som er bevidst tilgængelig.
Den er en simpel, ikke-refleksiv fornemmelse, som oftest kommer til syne gennem følelse og humør, men er altid til stede for bevidstheden.

Eksempler på affekt inkluderer glæde og sorg, afslappethed og anspændthed, energi og træthed.
Affekt kan være en isoleret, ren fornemmelse, men kan også indgå i en følelse eller et humør.

Et mere konkret eksempel er \textit{stolthed}, hvor man har det godt med sig selv.
Det at \textit{have det godt} er affekt, og \textit{med sig selv} er noget kognitivt.
Dette kvalificerer \textit{stolthed} som en følelse.

\subsection[Følelse]{Følelse\footnote{Engelsk: feeling}}
En typisk følelsesmæssig episode kan beskrives som en kompleks mængde af sammenhængende omstændigheder vedrørende et specifikt objekt, hvor dette objekt kan være en person, begivenhed eller genstand, uafhængigt af tid og virkelighed.

De sammenhængende omstændigheder dækker over a) affekt, b) synlig opførsel knyttet til følelsen (e.g. smil eller vidt-åbne øjne), c) opmærksomhed rettet mod katalysatoren, d) kognitiv vurdering af betydning og konsekvenser af katalysatoren, e) tillæggelse af begivenhederne der startede katalysatoren, f) oplevelsen af den pågældende følelse, g) neurale og endokrine ændringer som resultat af den pågældende følelse.

Eksempler på følelser er vrede, frygt, jalousi, stolthed og kærlighed.

\subsection[Humør]{Humør\footnote{Engelsk: mood}}
Noget som gør at humør skiller sig ud fra affekt og følelse, er at de typisk er længere-varende.
Humør er også ofte beskrevet som noget mere diffust og globalt, modsat specifikt.
Humør er den passende betegnelse for affektive tilstande der omhandler intet specifikt, eller alt -- generelt om verden.

Eksempelvis kan for en person som er bekymret (som er et humør), et objekt være fremtiden.
Ved en deprimeret person kunne objektet være personen selv som helhed.
For en irriteret person kan objektet være hvem eller hvad som helst.

Til forskel fra følelse, er katalysatoren for humør noget tidsmæssigt fjernt (e.g. en person kan vågne i dårligt humør som resultat af en hændelse den forrige dag).
Dette resulterer i at det ikke altid er klart hvordan et humør er opstået.

\paragraph{Stemningsleje} er et andet begreb ofte brugt i forbindelse med affektive lidelser.
Umiddelbart dækker det over det samme som humør, da det beskriver noget længerevarende.
På samme måde som man kan være i godt eller dårligt humør, kan man have hævet eller sænket stemningsleje.

\section{Forstyrrende og Ikke-Forstyrrende Dataindsamling}
Som det kan ses i titlen på rapporten, skelnes der overordnet mellem to typer af dataindsamling; forstyrrende og ikke-forstyrrende.

\subsection{Forstyrrende Dataindsamling}
Med forstyrrende menes der at der aktivt skal foretages noget af brugeren for at der kan leveres data om brugeren.
Det menes ikke som noget direkte negativt, blot at det er mere forstyrrende i forhold til dataindsamling som kan foregå i baggrunden uden bruger-interaktion.
Eksempler på forstyrrende dataindsamling kunne være at tage en selfie en gang i timen, eller mere relevant, at svare på spørgsmål vedr. stemningsleje.

Fordelen ved forstyrrende dataindsamling er at det er nemmere at få adgang til informationer om brugeren, og dennes tilstand, som man ikke umiddelbart kan via sensorer eller anden baggrunds dataindsamling.
Dette kunne være angående brugerens humør eller i hvor høj grad brugeren er sulten.
Ulempen ved denne type data er at den er ofte meget subjektiv; ved at spørge brugeren, og det er op til brugeren selv at svare, besvarelser manipuleres, hvilket kan give dirty data.

Den aktive data kunne suppleres med passiv (se den følgende sektion) data, for at afkræfte eller bekræfte brugerens svar.
Dette vil dog gå imod formålet med dette projekt, da det forventes at den aktivt indsamlet data er korrekt, således at den passivt indsamlet data kan knyttes til en ground truth (den aktive data).
Til gengæld kan der indsamles mere objektivt meta-data omkring brugerens besvarelser/interaktioner; såsom besvarelses-tid, antal skift af valg eller annullering.

\subsection{Ikke-Forstyrrende Dataindsamling}
Modsat den aktive dataindsamling, kræver den ikke-forstyrrende (passive) ingen interaktion med brugeren.
Dette gør at brugeren i mindre grad er påvirket af systemet, eller omvendt, og derfor vil den indsamlet data være mindre subjektiv.
Eksempler på ikke-forstyrrende dataindsamling kunne være bevægelsesmønstre via GPS eller tælling af SMSer og opkald.

Som sagt vil den passive data oftest være mere objektiv end den aktivt indsamlet data.
Dette gør at denne type data vil give et mere realistisk billede af en persons adfærd og stemningsleje.
Dette ligger tæt op ad målet for dette projekt, hvis det er muligt at give et objektivt og præcist billede af stemningsleje uden at dette kan manipuleres af brugeren.
