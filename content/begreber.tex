\section{Begreber}

\subsection{Affekt, følelse, humør}
For at videre kunne diskutere hvordan vi vil knytte sindstilstand og social aktivitet, er der nogle gængse begreber inden for psykologien der skal slås fast.
Affekt, følelse og humør (Engelsk: Affect, feeling, mood) er tre begreber som umiddelbart dækker over det samme, men hvor der indenfor psykologi-forskning er en forskel.
De definitioner som vil blive nævnt herefter er oversættelser baseret på \citet{ekkekakis}.

\subsubsection{Affekt}
Affekt beskriver en neurologisk tilstand, som er bevidst tilgængelig.
Den er en simpel, ikke-refleksiv fornemmelse, som oftest kommer til syne gennem følelse og humør, men er altid til stede for bevidstheden.

Eksempler på affekt inkluderer glæde og sorg, afslappethed og anspændthed, energi og træthed.
Affekt kan være en isoleret, ren fornemmelse, men kan også være indgå i en følelse eller et humør.

Et mere konkret eksempel er \textit{stolthed}, hvor man har det godt med sig selv.
Det at \textit{have det godt} er affekt, og \textit{med sig selv} er noget kognitivt.
Dette kvalificerer \textit{stolthed} som en følelse.

\subsubsection{Følelse}
En typisk følelsesmæssig episode kan beskrives som en kompleks mængde af sammenhængende omstændigheder vedrørende et specifikt objekt, hvor dette objekt kan være en person, begivenhed eller genstand, uafhængigt af tid og virkelighed.

De sammenhængende omstændigheder dækker over a) affekt, b) synlig opførsel knyttet til følelsen (e.g. smil eller vidt-åbne øjne), c) opmærksomhed rettet mod katalysatoren, d) kognitiv vurdering af betydning og konsekvenser af katalysatoren, e) tillæggelse af begivenhederne der startede katalysatoren, f) oplevelsen af den pågældende følelse, g) neurale og endokrine ændringer som resultat af den pågældende følelse.

Eksempler på følelser er vrede, frygt, jalousi, stolthed og kærlighed.

\subsubsection{Humør}
Noget som gør at humør skiller sig ud fra affekt og følelse, er at de typisk er længere-varende.
Humør er også ofte beskrevet som noget mere diffust og globalt, modsat specifikt.
Humør er den passende betegnelse for affektive tilstande der omhandler intet specifikt, eller alt -- generelt om verden.

Eksempelvis kan for en person som er bekymret (som er et humør), et objekt være fremtiden.
Ved en deprimeret person kunne objektet være personen selv som helhed.
For en irriteret person kan objektet være hvem eller hvad som helst.

Til forskel fra følelse, er katalysatoren for humør noget tidsmæssigt fjernt (e.g. en person kan vågne i dårligt humør som resultat af en hændelse den forrige dag).
Dette resulterer i at det ikke altid er klart hvordan et humør er opstået.
