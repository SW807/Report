\chapter{Begreber}
For at opnå en fælles forståelse for relevante psykologiske begreber vil de blive defineret i den følgende sektion.
Derefter vil der blive præsenteret en række begreber introduceret af projektgruppen, som er relateret til projektets emne.

\section[Affektive fænomener]{Affektive fænomener \footnote{Engelsk: affective phenomena}}\label{begreber:affektive}
Dette afsnit introducerer de tre fænomener affekt, følelse og humør, hvor fællesbetegnelsen benævnes \textit{affektive fænomener}.
Denne introduktion, samt de efterfølgende definitioner, er baseret på oversættelser af \citet{ekkekakis}.

Affektive fænomener er et stort og komplekst område.
De tre førnævnte begreber er nært relateret, hvilket har resulteret i adskillige teorier.
Der er stor uenighed, indenfor psykologien, om hvad de tre begreber hver især dækker over, samt hvordan de hænger sammen.
I \citet{ekkekakis} siges det at det ikke er muligt at skabe en dyb nok forståelse for dette emne, uden at læse et væld af artikler.
Til gengæld gives denne korte introduktion (de følgende definitioner af fænomenerne), hvorefter det opfordres at undersøge de angivne kilder, samt foretage egen undersøgelse.
De præsenterede definitioner er vores opfattelse af begreberne og disse betydninger vil blive benyttet igennem rapporten.

\subsection[Følelse]{Følelse\footnote{Engelsk: feeling}}
En typisk følelsesmæssig episode kan beskrives som en kompleks mængde af sammenhængende omstændigheder vedrørende et specifikt objekt, hvor dette objekt kan være en person, begivenhed eller genstand, uafhængigt af tid og virkelighed.

De sammenhængende omstændigheder dækker over a) affekt, b) synlig opførsel knyttet til følelsen (e.g. smil eller vidt-åbne øjne), c) opmærksomhed rettet mod katalysatoren, d) kognitiv vurdering af betydning og konsekvenser af katalysatoren, e) tillæggelse af begivenhederne der startede katalysatoren, f) oplevelsen af den pågældende følelse, g) neurale og endokrine ændringer som resultat af den pågældende følelse.

Eksempler på følelser er vrede, frygt, jalousi, stolthed og kærlighed.\cite[p. 322]{ekkekakis}

\subsection[Humør]{Humør\footnote{Engelsk: mood}}
Noget som gør at humør skiller sig ud fra affekt og følelse, er at de typisk er længere-varende.
Humør er også ofte beskrevet som noget mere diffust og globalt, modsat specifikt.
Humør er den passende betegnelse for affektive tilstande der omhandler intet specifikt, eller alt -- generelt om verden.

Eksempelvis kan en person som er bekymret (hvilket er et humør) fremtiden være det 'objekt' der bliver bekymret om.
Ved en deprimeret person kunne 'objektet' være personen selv som helhed.
For en irriteret person kan 'objektet' være hvem eller hvad som helst.

Til forskel fra følelse, er katalysatoren for humør noget tidsmæssigt fjernt (e.g. en person kan vågne i dårligt humør som resultat af en hændelse den forrige dag).
Dette resulterer i at det ikke altid er klart hvordan et humør er opstået.\cite[p. 322]{ekkekakis}

\paragraph{Stemningsleje} er et andet begreb ofte brugt i forbindelse med affektive lidelser.
Umiddelbart dækker det over det samme som humør, da det beskriver noget længerevarende.
På samme måde som man kan være i godt eller dårligt humør, kan man have hævet eller sænket stemningsleje.

\subsection[Affekt]{Affekt\footnote{Engelsk: affect}}
Affekt beskriver en neurologisk tilstand, som er bevidst tilgængelig.
Den er en simpel, ikke-refleksiv fornemmelse, som oftest kommer til syne gennem følelse og humør, men er altid til stede for bevidstheden.

Eksempler på affekt inkluderer glæde og sorg, afslappethed og anspændthed, energi og træthed.
Affekt kan være en isoleret, ren fornemmelse, men kan også indgå i en følelse eller et humør.

Et mere konkret eksempel er \textit{stolthed}, hvor man har det godt med sig selv.
Det at \textit{have det godt} er affekt, og \textit{med sig selv} er noget kognitivt.
Dette kvalificerer \textit{stolthed} som en følelse.\cite[p. 322]{ekkekakis}

\section{Forstyrrende og ikke-forstyrrende dataindsamling}
Som det kan ses i titlen på rapporten, skelnes der overordnet mellem to typer af dataindsamling; forstyrrende og ikke-forstyrrende.
Disse to begreber introduceres af projektgruppen for at klart at kunne skelne mellem de to meget forskellige metoder til indsamling af data.

\subsection{Forstyrrende dataindsamling}\label{begreber::forstyrrende}
Med forstyrrende menes der at der aktivt skal foretages noget af brugeren, for at der kan leveres data om brugeren.
Det menes ikke som noget direkte negativt, blot at det er mere forstyrrende i forhold til dataindsamling som kan foregå i baggrunden uden bruger-interaktion.
Eksempler på forstyrrende dataindsamling kunne være at tage et billede af sig selv en gang i timen eller at svare på spørgsmål vedrørende sit stemningsleje.

Fordelen ved forstyrrende dataindsamling er at det er nemmere at få adgang til informationer om brugeren, og dennes tilstand, som man ikke umiddelbart kan via sensorer eller anden baggrunds dataindsamling.
Dette kunne være angående brugerens humør eller i hvor høj grad brugeren er sulten.
Ulempen ved denne type data er at den ofte er meget subjektiv. 
Ved at spørge patienten skal brugeren selv tage stilling til sin tilstand og det er derfor muligt for patienten at manipulere med svaret, både bevidst og ubevidst.
Til gengæld kan der indsamles mere objektivt meta-data omkring brugerens besvarelser/interaktioner; såsom besvarelses-tid, antal skift af valg eller annullering.

\subsection{Ikke-forstyrrende dataindsamling}\label{begreber::ikke_forstyrrende}
Modsat den aktive dataindsamling, kræver den ikke-forstyrrende ingen interaktion med brugeren.
Dette gør at brugeren i mindre grad er påvirket af systemet, eller omvendt, og derfor vil den indsamlede data være mindre subjektiv.
Eksempler på ikke-forstyrrende dataindsamling kunne være bevægelsesmønstre via GPS eller optælling af SMS'er og opkald.

Som sagt vil den ikke-forstyrrende indsamlede data oftest være mere objektiv end den forstyrrende indsamlede data.
Dette gør at denne type data vil give et mere realistisk billede af en persons adfærd og stemningsleje.
Dette ligger tæt op ad målet for dette projekt, hvis det er muligt at give et objektivt og præcist billede af stemningsleje uden at dette kan manipuleres af brugeren.

\section{Social aktivitet}
Social aktivitet kan forstås som mange ting.
For de fleste vil det umiddelbart være en fysisk aktivitet foretaget sammen med andre, eller som minimum en aktivitet med en fysisk tilstedeværelse blandt andre.

I forhold til dette projekt ses social aktivitet som værende al aktivitet der involverer interaktion med andre personer.
Dette dækker over det fysiske samvær som nævnt ovenover og den interaktion, der foregår med andre personer over digitale medier.

Eksempler på sociale aktiviteter, foretaget på smartphone, kunne være den traditionelle brug af en mobiltelefon; opkald og SMS.
Andre eksempler kunne være sociale medier som Facebook eller Twitter.
Det kunne også være multiplayer-spil som 'Wordfeud' eller 'Draw Something'.

\paragraph{Måling af social aktivitet på smartphone} 
Målingen af den sociale aktivitet kan foregå på et væld af måder.
Der kan måles både kvantitativt og kvalitativt.

Kvantitativt kan der ses på antal unikke sociale interaktioner i løbet af en dag.
Der kunne også ses på hvor lange disse interaktioner er.

Kvalitativt kunne der ses på hvilken slags aktiviteter der foretages.
Lidt sværere er det at få information om hvordan brugeren har oplevet aktiviteten; om den var behagelig/ubehagelig.
