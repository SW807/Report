\section{Opkald}\label{datasamling:opkald}
På Android-smartphones er det muligt at få adgang til opkalds-historikken.
Denne beskriver de indgående, udgående og mistede opkald\footnote{Et mistet opkald er et indgående opkald som ikke besvares}, samt kontaktoplysninger og varighed af det enkelte opkald.
Informationen vedrører altså således ikke samtalens indhold, men kun meta-information om de enkelte opkald.

Det antages at denne meta-information vil kunne anvendes til at analysere patientens normale brug af telefon-opkald.
Ved at tælle antallet af opkald, ville man kunne observere en ændring ved henholdsvis begyndende depression eller mani.
Ved depression antages det at antallet af opkald, især udgående, vil falde.
Tilsvarende kan det antages at patienten har flere mistede opkald end det er normalt for patienten.
Det antages også at længden af opkaldene vil være kortere end normalt.
Som benævnt i fokusgruppe-interviewet, vil ens netværk også mindskes, hvilket vil kunne ses på modtager/afsender på opkaldene, hvor antallet af forskellige kontaktpersoner vil forventes at falde.
Derimod vil man ved mani forvente en stigning i opkald, især udgående.
Der vil også være en stigning i antal forskellige udgående kontakter.

Det forventes derfor at der kan forefindes en sammenhæng mellem antallet af opkald (både ind- og udgående), disses varighed og patientens stemningsleje.

\paragraph{Begrænsninger}
Eftersom der kun arbejdes med meta-information om opkaldene, er det ikke direkte muligt at sige noget om kvaliteten af opkaldene.
Det vil sige at der ikke kan siges noget om hvor meget og hvad der bliver sagt i den enkelte samtale.