\chapter{Datakilder}\label{datasamling}
Der findes i dag en række muligheder for social interaktion ved hjælp af digitale medier.
Meget af denne information kan tilgås direkte på en Android baseret smartphone (forudsat at patienten giver adgangstilladelse hertil).
Andre informationer, som fx email og kalender, kan indsamles fra \textit{skyen}.
Det vil sige at selvom disse oplysninger ikke er direkte tilgængelige på telefonen kan den stadig anvendes til at indsamle dem.
Herved opnås det at megen forskellig information kan indsamles kun ved brug af én enhed, hvilket gør det nemmere at skabe overblik over de patientoplysninger man måtte have.

I dette afsnit gives en oversigt over nogle af de muligheder der er for indsamling af data ved hjælp af en Android baseret smartphone.
Da dette projekt søger at etablere en relation mellem den indsamlede information og patientens stemningsleje, må listen herunder anses for u-udtømmelig, eftersom det i skrivende stund ikke er muligt at liste netop de oplysninger der relaterer sig til patientens stemningsleje.

\paragraph{Patient tilstand}
Alle henvisninger til patienters adfærd kan antages som værende udenfor maniske eller depressive perioder.
Dette gøres da det er vigtigt at opdage ændring i adfærden så tidligt som muligt, da lige så snart en episode er indtruffet vil applikationen ikke længere kunne assistere \citefaelles{Afsnit 1.5}.
Det er derfor nødvendigt at etablere en norm for den enkelte patient, således at ændringer i netop dennes adfærd kan detekteres.

%For hver mulighed, skitser en mulig løsning, hvorfor det er relevant, muligheder, begrænsninger.

\section{Opkald}\label{datasamling:opkald}
På Android-smartphones er det muligt at få adgang til opkalds-historikken.
Denne beskriver de indgående, udgående og mistede opkald\footnote{Et mistet opkald er et indgående opkald som ikke besvares}, samt kontaktoplysninger og varighed af det enkelte opkald.
Informationen vedrører således ikke samtalens indhold, men kun meta-information om de enkelte opkald.

Det antages at denne meta-information vil kunne anvendes til at analysere patientens normale brug af telefon-opkald.
Ved at tælle antallet af opkald, ville man kunne observere en ændring ved henholdsvis begyndende depression eller mani.
Ved depression antages det at antallet af opkald, især udgående, vil falde.
Tilsvarende kan det antages at patienten har flere mistede opkald end det er normalt for patienten.
Det antages også at længden af opkaldene vil være kortere end normalt.
Som benævnt i fokusgruppe-interviewet, vil ens netværk også mindskes, hvilket vil kunne ses på modtager/afsender på opkaldene, hvor antallet af forskellige kontaktpersoner vil forventes at falde.
Derimod vil man ved mani forvente en stigning i opkald, især udgående.
Der vil også være en stigning i antal forskellige udgående kontakter.

Det forventes derfor at der kan forefindes en sammenhæng mellem antallet af opkald (både ind- og udgående), disses varighed og patientens stemningsleje.

\paragraph{Begrænsninger}
Eftersom der kun arbejdes med meta-information om opkaldene, er det ikke direkte muligt at sige noget om kvaliteten af opkaldene.
Det vil sige at der ikke kan siges noget om hvor meget og hvad der bliver sagt i den enkelte samtale.
\section{SMS/MMS beskeder}\label{datasamling:smsmms}
Ligesom opkalds-historik, er der i Android mulighed for adgang til SMS/MMS beskeder.
Adgangen består her i adgang til både indhold, kontaktoplysninger, samt afsendelses tidspunkt.
Beskeder læses fra forskellige datasæt\footnote{Indbakke, udbakke, sendte beskeder, etc.} på telefonen.
Heraf er det implicit givet hvilke beskeder der er afsendt fra telefonen og hvilke der er modtagne.

Overvejelser vedrørende mulighederne for brug af opkald gør sig også gældende for beskeder.
Det vil altså sige, at man i forbindelse med en depression antager at der vil ske et fald i antallet af sendte beskeder, samt i antallet af forskellige kontakter.
Tilsvarende overvejelser gør sig gældende i forbindelse med mani.

Yderligere er det for beskeder muligt at analysere på beskeders indhold.
Eksempelvis kan der udføres sentiment-analyse for at forsøge at bestemme brugerens stemningsleje ud fra indholdet af en eller flere beskeder.

\paragraph{Begrænsninger}
Da beskeder ikke aktivt skal accepteres af brugeren på samme måde som et opkald, vil det ikke direkte være muligt at analysere på antallet af ubesvarede beskeder.
Som et alternativ hertil vil man kunne fokusere på at bestemme normer for svartid for den enkelte patient.

\section{Lokation}
Via smartphone, eller wearable som understøtter dette, er det muligt at forespørge på patientens nuværende lokation.
Dette gøres typisk ved hjælp af en GPS enhed i telefonen.
Informationen herfra er som minimum længdegrad, breddegrad samt et mål for præcisionen af disse værdier.
Enkelte enheder kan desuden angive yderligere information som fx højde.
Derudover kan flere forespørgsler kombineres for derved at kunne angive retning, hastighed og acceleration.

Nogle patienter vil deltage i sociale aktiviteter, såsom sport eller anden klub-aktivitet, fest og generel social aktivitet med familie eller venner.
Over tid ville man dermed kunne identificere nøgle-lokationer (fx. hjem, skole/arbejde, sportsklub, venner), da de ville være tilbagevendende.
For at bestemme nøgle-lokationer, kan patienten forespørges for korrekt at identificere disse.
Evt. kunne man få bruger-input til præcist at identificere nogle af disse lokationer.

Af fokusgruppe-interviewet fremgik det desuden at patienter med depression i høj grad søger at anvende en fast daglig struktur for bedre at være i stand til at administrere deres sygdom.
Altså vil ændringer i disse være tegn på en problematik.

Ved at skelne mellem sociale og ikke-sociale lokationer som for eksempel patientens hjem, vil man kunne registrere ændringer i den sociale aktivitet.
Ud fra den viden der er dannet i fællesrapporten, vil der for patienter på vej ind i en depression være et fald i social aktivitet.
For patienter på vej ind i en mani vil der omvendt være en stigning.

\paragraph{Begrænsninger}
Ved GPS-lokation alene, er det ikke direkte muligt at sige om en aktivitet er social eller ej.
Der vil være situationer hvor man er i ens hjem en hel weekend, men muligvis har besøg af venner eller familie.
Informationen fra GPS vil derfor i højere grad virke som et støttende værktøj til andre sensorer frem for en direkte informationskilde vedrørende social aktivitet.
\section{Applikations-brug}
I nyere versioner af Android \footnote{Android 5.0 Lollipop} er det muligt at få adgang til brugs-statistik for alle installerede applikationer.
Her er det muligt at forespørge på forskellige tidsintervaller, hvilket resulterer i en liste af applikationer brugt i tidsintervallet.
For hver applikation vil der være informationer omkring totalt tidsforbrug i millisekunder, tidligste og seneste brug (indenfor angivet interval).

Med denne information er det muligt at se hvor meget tid der bruges på applikationer med socialt indhold, som for eksempel Facebook, Skype e.l.

\paragraph{Begrænsninger}
Det er ikke muligt at se hvad de forskellige apps har været brugt til, men kun i hvor lang tid.
For eksempel vil man for Facebook-app'en ikke kunne se om der er brugt tid på at skrive til andre, læse nyheder eller spille spil.
\section{Kalender}
I Android er der en centraliseret adgang til kalender-oplysninger for alle kalendere telefonen har kendskab til.
Det vil sige at alle kalender-oplysninger kan tilgås \'et sted på en Android telefon.
Ved at lade sine online kalendere synkronisere med Android telefonen kan disse også tilgås via samme adgangsmekanisme.
Herved kan der opnås adgang til eventuelle kalendere på andre platforme.

Aftaler i kalendere kan give en indikation om planlagt social aktivitet, enten ved at patienten, for den enkelte aftale, angiver lokationen eller ved analyse af aftalens titel og eventuelle beskrivelse.
Ved at indsamle information om hvor de enkelte aftaler vil foregå og sammenholde dette med information om patientens faktiske lokation, vil det være muligt at vurdere om patienten deltager i sine planlagte aktiviteter.
Det vil her være nødvendigt at foretage yderligere analyse af aftalernes titel for at fastslå om disse er af social karakter.

Brugen af kalendere kan variere meget, og det vil derfor være nødvendigt at etablere en norm for den enkelte patient.
Herefter vil det forsøges, tilsvarende andre datakilder, at bestemme hvorvidt patienten afviger fra denne normale adfærd.
Ved mindre aktivitet end kalenderen foreskriver, kan patienten være på vej ind i en depression.
Tilsvarende kan forøget aktivitet være en indikator for en manisk periode.

\paragraph{Begrænsninger}
Nogle begivenheder vil være sæsonbetonede og en egentlig norm kan derfor være svær at opbygge på kort sigt.
Der kan desuden være tvetydige aftaler, som fx fødselsdage, hvor det vil være uklart om aftalen blot skal minde om dagen eller der er tale om en egentlig aftale.
\section{E-mail}
Som et led i indsamling af kommunikation, vil det være naturligt at analysere de emails en patient sender og modtager.
Ved indsamling af emails gælder de samme muligheder for data indsamling som ved sms og mms beskeder.
Dog vil emails typisk indeholde mere information end sms og mms beskeder.
Derfor vil det være oplagt at lægge yderligere fokus på analyse af indholdet i disse.

En yderligere fordel ved analyse af email er at det giver en indsigt i patientens digitale færden, udover direkte brug af mobiltelefonen.
Sendte og modtagne mails kan synkroniseres på tværs af enheder ved hjælp af IMAP protokollen, og dermed kan man opnå større indsigt i patientens kommunikationsvaner.

\paragraph{Begrænsninger}
Der er på Android ingen centraliseret lagring af email.
Det vil sige at der ikke er direkte mulighed for at få adgang til emails der ligger på telefonens email applikationer.
Hvis der ønskes adgang til email fra telefonen skal der implementeres håndtering af de primære email protokoller \footnote{POP3, IMAP og SMTP}.
Det betyder desuden at patienten vil skulle indtaste serveroplysninger for at dennes emails kan blive tilgængelig til analyse.
\section{Browser historik}
De hjemmesider en patient besøger (både på mobiltelefon og computer) kan være med til at beskrive vedkommendes stemningsleje.
Ved at have adgang til historikken over besøgte hjemmesider er det muligt at analysere hvilke informationer patienten søger.
Dette kunne eksempelvis ske ved at kigge på hvilke søgeord der anvendes.
Tilsvarende kan de besøgte hjemmesider analyseres for bestemte ord og sætninger.
Disse kan være prædefinerede og/eller oplyst af patienten.

Browser historik relaterer sig ikke direkte til social aktivitet, men kunne i stedet muligvis associeres direkte med stemningsleje.
Denne relation er kun spekulativ, men ville sandsynligvis kunne undersøges med metoder tilsvarende dem beskrevet i denne rapport.

\paragraph{Begrænsninger}
Information om hvilke hjemmesider der besøges i en bestemt browser vil kræve understøttelse for at indhente sådanne oplysninger fra de browsere den enkelte patient anvender.
Det kan derfor nemt betyde at mange browsere og muligvis formater ville skulle understøttes.
På samme måde som det er en fordel at kunne indsamle denne information på flere platforme, er det en øget udfordring da browsere ikke nødvendigvis opfører sig ens på tværs af platforme.
Det er pt. projektgruppen uvist hvilke browsere der understøtter muligheden for at indhente historik.