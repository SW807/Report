\section{Browser historik}
De hjemmesider en patient besøger (både på mobiltelefon og computer) kan være med til at beskrive vedkommendes sindstilstand.
Ved at have adgang til historikken over besøgte hjemmesider er det muligt at analysere hvilke informationer patienten søger.
Dette kunne eksempelvis ske ved at kigge på hvilke søgeord der anvendes.
Tilsvarende kan de besøgte hjemmesider analyseres for bestemte ord og sætningen.
Disse kan være prædefinerede og/eller oplyst af patienten.

Browser historik relaterer sig ikke direkte til social aktivitet, men kunne i stedet muligvis associeres direkte med sindstilstand.
Denne relation er kun spekulativ, men ville sandsynligvis kunne undersøges med metoder tilsvarende de beskrevet i denne rapport.

\paragraph{Begrænsninger}
Information om hvilke hjemmesider der besøges i en bestemt browser vil kræve understøttelse for at indhente sådanne oplysninger fra de browsere den enkelte patient anvender.
Det kan derfor nemt betyde at mange browsere og muligvis formater ville skulle understøttes.
På samme måde som det er en fordel at kunne indsamle denne information på flere platforme, er det en øget udfordring da browsere ikke nødvendigvis opfører sig ens på tværs af platforme.
Det er pt. projektgruppen uvist hvilke browsere der understøtter muligheden for at indhente historik.