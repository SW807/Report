\section{GPS}
Via smartphone, eller wearable som understøtter dette, er det muligt at forespørge på patientens nuværende lokation.
Dette resulterer, som minimum \mikkel{Som minimum? Hvad er der mere at hente?}, i en længdegrad, breddegrad og præcision.
Derudover kan flere forespørgsler kombineres til at sige noget om retning, hastighed og acceleration.

Nogle patienter vil deltage i sociale aktiviteter, såsom sport eller anden klub-aktivitet, fest og generel social aktivitet med familie eller venner.
Så ville man over tid kunne identificere nøgle-lokationer (fx. hjem, skole/arbejde, sportsklub, ven), da de ville være tilbagevendende.
Evt. kunne man få bruger-input til præcist at identificere nogle af disse lokationer.
Når først disse lokationer er bestemt, vil man kunne danne en model over den enkelte patients aktivitets-mønster.

Ved at skelne mellem sociale (venner, familie, sportsklub) og ikke-sociale lokationer (ens hjem, muligvis andre), ville man kunne registrere ændringer i den sociale aktivitet.
Ud fra den viden der er dannet i fællesrapporten, vil der for patienter på vej ind i en depression være et fald i social aktivitet.
For patienter på vej ind i en mani vil der omvendt være en stigning.

\paragraph{Begrænsninger}
Ved GPS-lokation alene, er det ikke direkte muligt at sige om en aktivitet er social eller ej.
Der vil være situationer hvor man er i ens hjem en hel weekend, men muligvis har besøg af venner eller familie.
Omvendt kunne man også tage til en familiebegivenhed, hvor man ikke nødvendigvis snakker eller på anden måde socialiserer med nogen.