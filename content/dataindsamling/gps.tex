\section{Lokation}
Via smartphone, eller wearable som understøtter dette, er det muligt at forespørge på patientens nuværende lokation.
Dette gøres typisk ved hjælp af en GPS enhed i telefonen.
Informationen herfra er som minimum længdegrad, breddegrad samt et mål for præcisionen af disse værdier.
Enkelte enheder kan desuden angive yderligere information som fx højde.
Derudover kan flere forespørgsler kombineres for derved at kunne angive retning, hastighed og acceleration.

Nogle patienter vil deltage i sociale aktiviteter, såsom sport eller anden klub-aktivitet, fest og generel social aktivitet med familie eller venner.
Over tid ville man dermed kunne identificere nøgle-lokationer (fx hjem, skole/arbejde, sportsklub, venner), da de ville være tilbagevendende.
For at bestemme nøgle-lokationer, kan patienten forespørges for korrekt at identificere disse.
Evt. kunne man få bruger-input til præcist at identificere nogle af disse lokationer.

Af fokusgruppe-interviewet fremgik det desuden at patienter med depression i høj grad søger at anvende en fast daglig struktur for bedre at være i stand til at administrere deres sygdom.
Altså vil ændringer i disse være tegn på en problematik.

Ved at skelne mellem sociale og ikke-sociale lokationer som for eksempel patientens hjem, vil man kunne registrere ændringer i den sociale aktivitet.
Ud fra den viden der er dannet i fællesrapporten, vil der for patienter på vej ind i en depression være et fald i social aktivitet.
For patienter på vej ind i en mani vil der omvendt være en stigning.

\paragraph{Begrænsninger}
Ved GPS-lokation alene, er det ikke direkte muligt at sige om en aktivitet er social eller ej.
Der vil være situationer hvor man er i ens hjem en hel weekend, men muligvis har besøg af venner eller familie.
Informationen fra GPS vil derfor i højere grad virke som et støttende værktøj til andre sensorer frem for en direkte informationskilde vedrørende social aktivitet.