\section{Kalender}
I Android er der en centraliseret adgang til kalender-oplysninger for alle kalendere telefonen har kendskab til.
Det vil sige at alle kalender-oplysninger kan tilgås \'et sted på en Android telefon.
Ved at lade sine online kalendere synkronisere med Android telefonen kan disse også tilgås via samme adgangsmekanisme.
Herved kan der opnås adgang til eventuelle kalendere på andre platforme.

Aftaler i kalendere kan give en indikation om planlagt social aktivitet, enten ved at patienten, for den enkelte aftale, angiver lokationen eller ved analyse af aftalens titel og eventuelle beskrivelse.
Ved at indsamle information om hvor de enkelte aftaler vil foregå og sammenholde dette med information om patientens faktiske lokation, vil det være muligt at vurdere om patienten deltager i sine planlagte aktiviteter.
Det vil her være nødvendigt at foretage yderligere analyse af aftalernes titel for at fastslå om disse er af social karakter.

Brugen af kalendere kan variere meget, og det vil derfor være nødvendigt at etablere en norm for den enkelte patient.
Herefter vil det forsøges, tilsvarende andre datakilder, at bestemme hvorvidt patienten afviger fra denne normale adfærd.
Ved mindre aktivitet end kalenderen foreskriver, kan patienten være på vej ind i en depression.
Tilsvarende kan forøget aktivitet være en indikator for en manisk periode.

\paragraph{Begrænsninger}
Nogle begivenheder vil være sæsonbetonede og en egentlig norm kan derfor være svær at opbygge på kort sigt.
Der kan desuden være tvetydige aftaler som fx fødselsdage, hvor det vil være uklart om aftalen blot skal minde om dagen eller der er tale om en egentlig aftale.