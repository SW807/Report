\section{Applikations-brug}
I nyere versioner af Android\footnote{Android 5.0 Lollipop} er det muligt at få adgang til brugs-statistik for alle installerede applikationer.
Her er det muligt at forespørge på forskellige tidsintervaller, hvilket resulterer i en liste af applikationer brugt i tidsintervallet.
For hver applikation kan der indhentes informationer om totalt tidsforbrug i millisekunder, tidligste og seneste brug (indenfor et angivet interval).

Denne information kan bruges til at bestemme en norm for brugen af sociale applikationer.
Herved forstås applikationer hvis primære formål er at etablere kommunikation.
Altså eksempelvis Facebook, Twitter, Skype, etc.
Ved at etablere en norm for brugen af sociale applikationer kan det undersøges hvorvidt denne brug senere ændres i en sådan grad at det kan antages at patienten er på vej ind i en depression eller mani.
Parallelt til opkald og beskeder, forventes brugen af de social applikationer at falde i forbindelse med depression og stige i forbindelse med mani.

\paragraph{Begrænsninger}
Denne metode giver udelukkende adgang til information om hvor lang tid en applikation anvendes.
Det er altså ikke muligt at se hvad de forskellige applikationer anvendes til.
Eksempel vil man med Facebooks applikation ikke kunne se om patienten har kommunikeret med andre personer, læst nyheder eller spillet spil.

Det skal dog nævnes at denne løsning er generel og dækker over alle applikationer.
Information om hvilken brug en patient gør af en enkelt social platform kan muligvis indhentes fra den enkelte platform.
Denne problemstilling er ikke blevet undersøgt af projektgruppen.