\section{Social indflydelse på mental tilstand}

For at kunne benytte vores logning af sociale aktiviteter er det nødvendigt at 
vide hvordan social aktivitet har indflydelse på patienters mentale tilstand.

Der er i psykologien to modeller der modellerrer sammenhængen mellem social aktivitet og mental tilstand, "stress buffering" modellen og "main effects" modellen\cite{socialties}.

\paragraph{Stress buffering modellen} indikerer at social support modulerer eller i bedste fald forhindrer at stressfulde begivenheder har en skadende effekt på den mentale tilstand for individer der allerede er under stress \cite[p.~460]{socialties}.
Ved en stressfyldt begivenhed vil både den opfattede sociale support og den modtagne sociale support have indflydelse på hvordan individet klarer sig igennem begivenheden.\cite[p.~460]{socialties}.

\paragraph{Main effects modellen} postulerer at al social indflydelse har en effekt på individets mentale helbred.
Integration i sociale sammenhænge kan derfor indvirke positivt på den psykologiske tilstand for individet, hvilket vil have en positiv effekt på den mentale tilstand.\cite[p.~459]{socialties}.
 
\paragraph{Social baseline}
Under mødet med Jørgen Aagaard blev det gjort meget klart at det var meget vigtigt at betragte ændringer i adfærd, og ikke sammenligne med en forudbestemt værdi. 
Det er derfor nødvendigt at konstruere en model der finder en baseline for patientens sociale aktivitet, og derefter sammenligne adfærden med denne adfærd.
\stefan{reference til sektion der er i den anden rapport?}
