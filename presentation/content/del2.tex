\newcommand{\gsms}{
\draw [line width=1pt, magenta] plot [smooth] coordinates {(0,0) (0.2563,0.367) (0.4765,0.1869) (0.6004,0.6964) (0.777,0.0586) (0.9044,0.3222) (1.0,0.4321)};
\node[anchor=west] at (1.0,0.4321) {SMS};
}
\newcommand{\gopkald}{
\draw [line width=1pt, fom] plot [smooth] coordinates {(0,0) (0.2325,0.2839) (0.4604,0.2355) (0.6001,0.7629) (0.7638,0.0278) (0.8912,0.4276) (1.0,0.5507)};
\node[anchor=west] at (1.0,0.5507) {Opkald};
}
\newcommand{\gstem}{
\draw [line width=1pt, orange] plot [smooth] coordinates {(0,0) (0.2571,0.3322) (0.463,0.275) (0.6072,0.8024) (0.7777,0.0343) (0.9314,0.2407) (1.0,0.3178)};
\node[anchor=west] at (1.0,0.3178) {Stemningsl.};
}

\newcommand{\mygraph}[1]{\definecolor{fom}{RGB}{0,153,139}
\begin{tikzpicture}[x=7cm,y=4cm]
\draw[thick, lightgray, ->] (0,0) -- (0,1);
\draw[thick, lightgray, ->] (0,0) -- (1,0);

#1
 
\node[label={[label distance=0.0cm,text depth=-1ex]center:Tid}] at (0.5,-0.05) {};
\end{tikzpicture}}
\newcommand{\pbb}[1]{\parbox[m]{3cm}{\vspace{.5\baselineskip} \raggedright {#1} \vspace{.5\baselineskip}}}
\section{Diskussion}
\subsection{Survey-modulet}
\begin{frame}{Modul forbedringer}
\begin{itemize}
\item Manglende besvarelse af spørgsmål:
\end{itemize}
\begin{center}
\begin{tabular}{|l| >{\raggedright}m{3cm} | c |}\cline{2-3}
\multicolumn{1}{l|}{} & Aktiv & Passiv \\\hline
Udeladelse & \pbb{Patient vælger at han ikke vil besvare.} & \pbb{Patienten undlader besvarelse.} \\\hline
Udsættelse & \pbb{Patient vælger at udsætte besvarelse.} & \pbb{Patienten undlader besvarelse, og spørges igen senere.} \\\hline
\end{tabular}
\end{center}
\end{frame}

\begin{frame}{Modul forbedringer}
\begin{itemize}
\item Tidspunkt for spørgsmål
\begin{itemize}
\item Reduktion i antallet af samtidige spørgsmål.
\item Variation af spørgsmål.
\item Spørgsmåls-sæt besvares over tid og kontinuert.
\item Spørgsmål besvares på samme tid af dagen.
\end{itemize}
\item Størrelse på tekstfelt\\
\includegraphics[width=0.37\textwidth]{tekst}
\item Farver på svarmuligheder
\end{itemize}
\end{frame}

\begin{frame}{Ikke-eksperimentiel}
Information fra survey-modulet kan anvendes som datakilde på lige fod med andre datakilder.
Dette lader patienten følge udviklingen i egne besvarelser af fx:
\begin{itemize}
\pause
\item Dagbog
\begin{itemize}
\item Daglige besvarelser
\item Ikke analytisk
\end{itemize}
\pause
\item Stemningsleje\\

\mygraph{\gstem{}}
\end{itemize}
\end{frame}

\subsection{Eksperimenter}
%(Sammenligning på tværs af sygdomme og Sammenligning på tværs af sensorer/analyser)

\begin{frame}{Social aktivitet}
\only<1>{\mygraph{\gstem{}}}
\only<2>{\mygraph{\gsms{}\gstem{}}}
\only<3>{\mygraph{\gsms{opacity=0.15}\gopkald{}\gstem{}}}
\only<4>{\mygraph{\gsms{}\gopkald{}\gstem{opacity=0.5}}}
\only<5>{\mygraph{\gsms{dashed, opacity=0.7}\gopkald{dashed, opacity=0.7}\gsocial{}\gstem{opacity=0.2}}}
\only<6>{\mygraph{\gsocial{}\gstem{}}}
\end{frame}

\subsection{Validering}
\begin{frame}{Indsamling af data}
Data-kilder vurderes på følgende tre kriterier:
\begin{itemize}
\item Pålidelighed
\item Præcision
\item Frekvens
\end{itemize}\pause
Begrænsninger i kilder kompenseres ved at kombinere flere:
\begin{itemize}
\item Stemningsleje på mobil: Ofte men upålideligt
\item Stemningsleje fra psykolog: Sjældent men pålideligt
\end{itemize}
\end{frame}

\begin{frame}{Stemningsleje og social aktivitet}
Der findes ingen \textit{ground truth} til validering.
\begin{itemize}
\item Stemningsleje kan bestemmes af psykolog/psykiater
\begin{itemize}
\item Indsamlet information kan sammenholdes hermed.
\item Kontinuert indsamling $\rightarrow$ kontinuert udvikling.
\end{itemize}\pause
\item Social aktivitet
\begin{itemize}
\item Ingen kendt skala.
\item Ingen kendt metode for fagfolk.
\item Bestemmes ved relation mellem forskellige kilder på social aktivitet.
\item Kontinuert indsamling $\rightarrow$ kontinuert udvikling.
\end{itemize}
\end{itemize}
\end{frame}

\subsection{Åbne problemstillinger}
\begin{frame}{Åbne problemstillinger}
4 åbne problemstillinger introduceret:
\begin{itemize}
\item \textit{Hvordan skal de indsamlede data analyseres?}
\only<2>{\begin{itemize}\item Afhænger af det data der skal undersøges.\end{itemize}}

\item \textit{Hvordan oplyser vi patienten, om de informationer der er i hæftet, på en mobil platform?}
\only<3>{\begin{itemize}\item Kort guide associeret med hvert spørgsmål.\end{itemize}}
%VIS HJÆLPE-FUNKTION

\item \textit{Hvordan præsenteres HAM-D på en mobil platform uden en interviewer?}
\only<4>{\begin{itemize}\item Interview udføres af chatbot, særlig trænet til denne type opgave.\end{itemize}}
%DET KAN MAN IKKE :P
%FORDI SVARET SKAL FORTOLKET (AF FAGFOLK)

\item \textit{Hvornår skal spørgsmål stilles for at få det bedste billede af patientens stemningsleje?}
%\only<5>{\begin{itemize}\item Kort guide associeret med hvert spørgsmål.\end{itemize}}
%DET ER DET SAMME IGEN
%SVINGNING I LØBET AF DAGEN

\end{itemize}
\end{frame}
